%%%% Class %%%%

% Class of the document: KOMA Script Article
%   Options: 12 points font, A4 paper, bibliography in ToC
\documentclass[a4paper,fontsize=11pt,bibliography=totoc]{scrartcl}

%%%% Preamble %%%%
% Packages inclusions
%\usepackage[top=0.75in,bottom=0.8in,left=0.75in,right=0.75in]{geometry}								% Custom margins (comment out if not needed)
\usepackage[english]{babel}				% Language settings
\usepackage{lmodern}					% Use Latin Modern font
\usepackage[nointegrals]{wasysym}
\usepackage{microtype}					% Enhanced typesetting (better reading)
\usepackage{graphicx}					% Enhanced graphics processing
\usepackage{url}
\usepackage{breakurl}
\usepackage[table,hyperref,x11names,svgnames]{xcolor}				% Allows coloring tables
\usepackage[section]{placeins}			% Ensure float elems. inside their section
\usepackage{booktabs}					% Beautiful, "pro" tables
\usepackage{adjustbox}					% Boxing capabilities for images and text
\usepackage{multirow}					% Enables tabular cells spanning multiple rows
\usepackage{array}					% Enables custom table column defintions
\usepackage[toc,page]{appendix}			% Appendices title customisation
\usepackage{enumitem}					% Customized enumerations and itemizes
\usepackage[version=3]{mhchem}			% Allows inclusion of chemical formulas and eq.
\usepackage{amssymb}
\usepackage[ruled,vlined]{algorithm2e} % Algorithms and pseudocode
\usepackage{todonotes}					% Useful to put TODO side notes
\usepackage[noblocks]{authblk}			% Allows authors with affiliations
\usepackage{lastpage}					% Allows referencing the last page
\usepackage[style=british]{csquotes}    % Allows block quotations
\usepackage{titlesec}
\usepackage{nameref}	% Allows references by name ("Section 5" vs "Conclusions")
\usepackage{fancyhdr}					% Fancy headers and footers
\usepackage{multicol}
\usepackage{lipsum}
\usepackage[
	backend=biber,
	natbib=true,
	bibstyle=authoryear,
	citestyle=authoryear,
	sorting=nyt,
	block=space,
	hyperref=true,
	dashed=false
]{biblatex}
\usepackage{hyperref}
\hypersetup{
  unicode=true,
 	pdfpagemode={UseOutlines},
 	pdfstartview={FitV},
	bookmarks=true,
	bookmarksopen=true,
	bookmarksopenlevel=0,
	bookmarksnumbered=true,
	breaklinks=true, % to have links breaking among lines
	hypertexnames=true,
	plainpages=false,
	hidelinks=false,
	colorlinks=true,
	citecolor=Turquoise4,
	filecolor=black,
	linkcolor=DeepSkyBlue4,
	urlcolor=RoyalBlue3,
	anchorcolor=RoyalBlue3
}

% Command definitions
\makeatletter

% useful cross referencing commands
\newcommand{\fref}[1]{Figure~\ref{#1}}
\newcommand{\tref}[1]{Table~\ref{#1}}
\newcommand{\eref}[1]{Equation~\ref{#1}}
\newcommand{\cref}[1]{Chapter~\ref{#1}}
\newcommand{\sref}[1]{Section~\ref{#1}}
\newcommand{\aref}[1]{Appendix~\ref{#1}}
\newcommand{\alref}[1]{Algorithm~\ref{#1}}
\newcommand{\procref}[1]{Procedure~\ref{#1}}

\newcommand\frontmatter{%
	\cleardoublepage
	\pagenumbering{roman}}

\newcommand\mainmatter{%
	\cleardoublepage
	\pagenumbering{arabic}}

\newcommand\backmatter{%
	\cleardoublepage
}

\renewcommand\Authands{ and }

\newcommand{\norm}[1]{\lvert #1 \rvert}

\newcommand{\titlemake}[1]{%
		\begin{center}
%			\begingroup
				\Large\sffamily\bfseries{#1}
%			\endgroup
		\end{center}
}

\newcommand{\subtitlemake}[1]{%
	\begin{center}
		\begingroup
			\large\sffamily{#1}
		\endgroup
	\end{center}
}

\newcommand{\old}[1]{\textcolor{Red2}{#1}}

\makeatother

\newenvironment{checklist}{%
  \begin{list}{}{}% whatever you want the list to be
  \let\olditem\item
  \renewcommand\item{\olditem[$\Box$] }
  \newcommand\checkeditem{\olditem[$\CheckedBox$] }
}{%
  \end{list}
}

% Koma Script font modifications
\setkomafont{author}{\small}
\setkomafont{date}{\scshape}
\addtokomafont{section}{\large}
\addtokomafont{subsection}{\normalsize}
\addtokomafont{subsubsection}{\small}
\addtokomafont{caption}{\small}
\setlength{\abovecaptionskip}{2pt plus 0pt minus 2pt}
\setlist{nosep}
\renewcommand*{\bibfont}{\footnotesize}

% Formatting for paragraphs (indentation and space after par.)
\setlength{\parskip}{0.5\baselineskip}%
\setlength{\parindent}{1em}%

% Column separation
\setlength{\columnsep}{1.5em}

% Reduce space after sections
\RedeclareSectionCommands[
	afterskip=0.2pt
]{section,subsection,subsubsection}
%\titlespacing{\section}{0pt}{-0.25\parskip}{-0.5\parskip}
%\titlespacing{\subsection}{0pt}{-0.25\parskip}{-0.6\parskip}
%\titlespacing{\subsubsection}{0pt}{-0.5\parskip}{-0.75\parskip}

% Configuration of the blockquotes paragraphs
\SetBlockThreshold{0} % all blockquotes of 1 or more lines are treated as blocks

%\addbibresource{references.bib}

%%%% Document %%%%

\begin{document}
% Page style set to fancy to get customized headers
\pagestyle{fancy}
\fancyhf{} % clear header and footer
\rhead{\footnotesize \today}
\lhead{\footnotesize David Martínez Rodríguez}
\chead{\footnotesize Opposition Report}
\rfoot{\footnotesize \thepage}

\titlemake{ % \titlemake[1] is a custom command -- look in the preface!
Harvesting and utilizing beach cast on Gotland}
\subtitlemake{Opposition report on Filip Dessle's MSc thesis}\vspace{0.5em}
%\begin{multicols}{2}
The thesis work by Filip Dessle assesses the potential of harvesting beach cast in the shores of Gotland as a way to mitigate coastal eutrophication in this Baltic island. The analysis is, however, mostly centred on three different strategies of beach cast utilisation that the author argues would take the mitigation effects even further: (1) biogas production with digestate recovery for fertilisation, (2) fertilisation of food crops and (3) fertilisation of \emph{Salix} sp. plantations. Although based on the assessment of those strategies, the main result of the thesis is not conveyed in the clearest way and, actually, it is included in the Discussion section. A final and separate subsection, within the Results section, with a summary of the potential contribution to eutrophication mitigation by each of the strategies would help to wrap the results around the initially stated aim.

With regards to the actual results, it is very positive that the author has dealt with a wide range of aspects other than eutrophication itself (social and economic assessments): it shows that a systems perspective was guiding the evaluation of the situation and the alternative solutions. In this respect, the thesis truly fits the sustainability studies scholarship. Even though the analysis of the utilisation options is highly relevant, because it goes beyond eutrophication mitigation, the results seem to be a bit outside of focus, with respect to the objectives set in the \emph{Aims and objectives} section. However, they are not misaligned with the aims. Therefore, it is suggested that a re-formulation of the objectives is made to fit both (1) the broader stated aim and (2) the set of actual results. Alternatively, the results from each of the beach cast use strategies could be put in context with respect to eutrophication mitigation more explicitly.

A further point of constructive critique is the Methodology section. While it describes the procedural approach to achieve the results, it does so in a manner that feels too sequential. Instead, the suggestion for improvement is that the methods are explained separately and, for each technique, relate it to the set of results it contributes to. This is, instead of stating for each result the methods used, state, for each method, the results it provides. This approach would clarify the actual methodology used throughout the thesis, which seems too loose in the current formulation. Moreover, higher emphasis should be put in \emph{describing} the surveys and interviews that the author conducted\footnote{For example, no reference in the Methods section is made to the appendix where the surveys are reproduced, and no explanation is given on how and why where they conducted.}. Currently, they are embedded in the description in a way that makes them go unnoticed. In addition, the Methodology section should not contain numerous source citations---if literature is used as a data input for a result, then the method description should just state this fact. Finally, there is no clear connection between the last method (and result) used in the thesis (the SWOT analysis) and the actual objectives of the study. Either the methods or the objectives need to be reformulated to accommodate this last contribution.

The conclusions of the thesis, together with the discussion, are relevant, concisely written and, most importantly, connected to the results (coherent). The acknowledgement of the limitations in eutrophication mitigation from beach cast harvesting is consistent with the results and an important contribution as a whole. Moreover, the author also pinpoints, through the SWOT analysis first and the Discussion later, the obstacles to a feasible implementation of either of the strategies. The author also provides with some hints on how to overcome some of those obstacles, which in some cases involves further research and, in some others, a market integration of the value that beach cast harvesting would entail. In this respect, a more thorough discussion could be given by the author on the mechanisms to include the externalities and ecosystem services values that would open the door to the economic feasibility of harvesting beach cast.

An idea/question emerges throughout the thesis, with respect to eutrophication mitigation. Even though it is outside of the scope, the author could benefit from a discussion on the advantages and drawbacks of algae cultivation for nutrient unloading in the Baltic sea. This is, a discussion contrasting this alternative and the thesis targeted technique (beach cast harvesting) would be a great addition. Such a discussion might spark a debate regarding the most important motivation behind beach cast harvesting for the local communities: eutrophication mitigation, the need for (renewable) and cost-efficient biofertilisers or the recovery of the high recreational value that beaches in Gotland suppose (especially for the tourism industry).

Finally, regarding the outline of the thesis, the Background section could actually be incorporated to the Introduction. If it is to remain a separate section, it should, anyway, be placed \emph{before} the Methods, because it currently suggests that the theoretical background information is part of the results of the thesis. Besides this issue, the thesis is well structured, following a straight narrative that facilitates understanding the topic, the research that was performed and its final results. However, on the side of formatting, the thesis should be re-worked a bit: captions on tables should be moved to the top, the table of contents should be revised to avoid ordering issues, the text should be adjusted both to the left and right and, finally, several paragraphs along the document should be split up to avoid too large descriptions or argumentations.

%\printbibliography
%\end{multicols}
\end{document}