\section{Specific research question}
\label{s:research-question}
The CERO framework (Climate and Economic Research in Organisations) assesses the potential of greenhouse gas (GHG) emissions reduction that organisations policies can deliver~\parencite{robert2009cero}. A natural extension of the framework, already performed by \textcite{robert2006stockholm2030,robert2016cero-regional}, is to scale up the process to a regional level, and to evaluate the aggregate potential of several policies implemented simultaneously. However, the synergistic effects of multiple measures are difficult to estimate and, moreover, the CERO framework is heavily focused on \ce{CO_2} emissions reductions alone. Although \textcite{robert2006stockholm2030} also calculated social costs cuts derived from a more sustainable transportation system, such estimations are not systematically included in the CERO model.

Given the complexity involved when evaluating simultaneous courses of action to reduce GHG emissions and the lack of assessment of additional potential benefits, this thesis addresses the following research question:
%
\blockquote{\textit{``How can transport policies and measures aimed at pollution mitigation be assessed under a holistic, synergetic perspective?''}}

\section{Aim and objectives}
\label{s:aim-objectives}
To address both identified gaps in the CERO framework (synergistic behaviour of policies and additional social benefits to airborne pollution mitigation), an ambitious aim is set for the thesis: \textit{to design a system-wide evaluation framework for urban transportation-related policies, especially those which target the reduction of GHG emissions. This framework should be able to capture the internal dynamics and interrelations between the core components of the system and the societal context in which it is embedded. Moreover, the framework (at least at a conceptual level) must be generalisable and transferable to different cities}.

In order to achieve the aim, a series of objectives are laid out to incrementally build and validate the evaluation framework:
%
\begin{enumerate}[label=(\alph*)]
	\item Identify the important variables affecting the transportation system in urban areas.
	\item Design a conceptual evaluation model for the transportation system.
	\item Expand the conceptual model to include other social and economic aspects.
	\item Quantify the relations within the model to improve later policy assessments.
	\item Integrate the model with the CERO framework.
	\item Test the validity of the model with regards to policy assessment.
\end{enumerate}

\section{Relevance}
\label{s:relevance}
Airborne pollution is one of the main causes of respiratory diseases and associated increase in morbidity in densely populated areas~\parencite{who2013review,vimercati2011airpollutionmorbidity}. The link from air pollution to both severe health problems and high traffic volumes is well known and thoroughly researched~\parencite{who2006air}. However, an unsustainable transportation system not only causes health issues, but massive costs in terms of reduced productivity, freight delay, increased energy and fuel consumption or vehicle losses and damages, for example~\parencite{lizeng2012costcongestion}. To address all these issues, policy packages or simultaneous enforcing of different policies are needed, because of the complexity involved in effectively reducing transportation impacts~\parencite[ch. 3, p. 45]{garcia2014travel}.

Policy assessment is, therefore, a key issue to develop, due to the difficulty of dealing with entire systems, their internal dynamics and the emergent systemic behaviour patterns, such as feedback loops, rebound effects and hidden causalities. In this regard, system dynamics can help capture such structures and cause-effect chains, thus playing a \textit{discoverer} role, while the use of participatory backcasting (as performed in the CERO framework) to design the policies that best fit the situation -- and then evaluated by the system dynamics framework -- is an example of the \textit{mediator} role that science plays in sustainability issues~\parencite{ozawa1996science}.

\section{System borders and delimitations}
\label{s:system-borders}
The boundaries of the system under analysis in the thesis are set around the transportation system and the \textit{primary} effects and variables that affect it. As explained in the \nameref{s:methods} section (\ref{s:methods}) below, an expansion of the initial system boundaries is planned in the model development phase. The borders expansion is meant to capture more complex system behaviour patterns, thus providing more useful insights to policy makers using the evaluation framework.

\section{Methods}
\label{s:methods}
Regarding the methodological approach of the thesis, the chosen tool for the modelling step (the main contribution of the study) is the \textit{causal loop diagram} (CLD). The models obtained using this technique are easy to understand by a majority of stakeholders, they quickly convey the interrelations of the main components of a (dynamic) system and, finally, they are useful to highlight the feedback structure of the modelled system~\parencite{ghosh2015dynamic}.

A more detailed plan of the methodology is depicted in the following enumeration, which matches the objectives presented in \sref{s:aim-objectives}:
%
\begin{enumerate}[label=(\alph*)]
	\item Perform a literature review to retrieve data on the transportation system components.
	\item Design a \textit{conceptual} CLD, linking different components in the transport system.
	\item Iteratively expand the system boundaries of the CLD by including other aspects that affect or are affected by the transport system, with the insights provided by the literature research.
	\item Using data from the literature review and from the database collected in the CERO project, fit as many equations as possible for the links in the model. The quantification is done in two steps:
	\begin{enumerate}[label=\roman*.]
		\item Identify and define the quantifiable links between \textit{core components} of the system.
		\item Quantify ``external'' links dealing with social issues, such as healthcare or accident costs, if enough data is available.
	\end{enumerate}
	\item Design and discuss with the relevant stakeholders a methodology to integrate the dynamic model in the policy assessment process of the CERO framework.
	\item Perform a validation test of the evaluation model in a case study in Barcelona (e.g. in the context of the GrowSmarter\footnote{The GrowSmarter project is a joint effort from companies, public authorities and researchers to implement pilot measures to achieve a more sustainable life in the cities of Stockholm, Cologne and Barcelona. Some of the measures address sustainable transportation with the goal to reduce \ce{CO_2} emissions by 60\%. More information can be found at: \url{http://www.grow-smarter.eu/home}} project).
\end{enumerate}

\section{Evidence}
\label{s:evidence}
The data used to perform the design of the evaluation framework will be based, primarily, on a literature review. A preliminary selection of this evidence is found in cost estimation studies performed by, e.g., \textcite{lizeng2012costcongestion,robert2006stockholm2030,trafikverket2016asek,mizutani2011estimatingcosts}. In addition, datasets developed within the CERO project along many years might be used, both to extract new links and concepts to add to the conceptual model and to quantify some of those relations.

\section{Personal justification}
\label{s:personal-justification}



\section{Generalisation and transferability}
\label{s:generalisation}
