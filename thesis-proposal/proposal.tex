\section{Specific research question}
\label{s:research-question}
The \textbf{C}limate and \textbf{E}conomic \textbf{R}esearch in \textbf{O}rganisations framework (CERO) assesses the potential of greenhouse gas (GHG) emissions reduction that organisations policies can deliver~\parencite{robert2009cero}. A natural extension of the framework, already performed by \textcite{robert2006stockholm2030,robert2016cero-regional}, is to scale up the process to a regional level, and to evaluate the aggregate potential of several policies implemented simultaneously. However, the synergistic effects of multiple measures are difficult to estimate and, moreover, the CERO framework is heavily focused on \ce{CO_2} emissions reductions alone. Although \textcite{robert2006stockholm2030} also calculated social costs cuts derived from a more sustainable transportation system, such estimations are not systematically included in the CERO model.

Given the complexity involved when evaluating simultaneous courses of action to reduce GHG emissions and the lack of assessment of additional potential benefits, this thesis addresses the following research question:
%
\blockquote{\textit{``How can transport policies and measures aimed at pollution mitigation be assessed under a holistic, synergetic perspective?''}}

\section{Aim and objectives}
\label{s:aim-objectives}
To address both identified gaps in the CERO framework (synergistic behaviour of policies and additional social benefits to airborne pollution mitigation), an ambitious aim is set for the thesis: \textit{to design a system-wide evaluation framework for urban transportation-related policies, especially those which target the reduction of GHG emissions. This framework should be able to capture the internal dynamics and interrelations between the core components of the system and the societal context in which it is embedded. Moreover, the framework (at least at a conceptual level) must be generalisable and transferable to different cities}.

In order to achieve the aim, a series of objectives are laid out to incrementally build and validate the evaluation framework:
%
\begin{enumerate}[label=(\alph*)]
	\item Perform a literature review on the most important concepts and variables affecting the transportation system in urban areas.
	\item Design a conceptual \textit{causal loop diagram} (CLD) linking different components in the transport system.
	\item Incrementally expand the system boundaries of the CLD by including other aspects that affect or are affected by the transport system (e.g. taxes, fuel subsidies, healthcare costs, etc.).
	\item 
	
	\item Describe the current situation with regards to air pollution in the city: its main causes and consequences.
	\item Analyse the potential for up-scaling the CERO framework to a regional level.
	\item Evaluate the usefulness of individual behaviour modelling with respect to policy at the municipal and regional levels.
	\item Perform a literature review on the most effective in-place regulations to tackle air pollution.
	\item Perform a stakeholder analysis with regards to transport and related regulation in the chosen area.
	\item Propose a strategy to implement the CERO (individual behaviour modelling and backcasting approach) framework in the context of Barcelona.
\end{enumerate}

\section{Relevance}
\label{s:relevance}

The most prominent origin of air pollution in the urban region of Barcelona is the transport system. The previously stated research question can be narrowed down by taking this fact into account. 

The most prominent contribution is that these measures are derived from a participatory backcasting process. Public authorities involvement in the 2006 article

the high levels of air pollutants in Barcelona and

(especially in terms of \ce{NO_X} and \ce{O_3})

\old{
\begin{itemize}
	\item Geographic and climate constraints.
	\item Geopolitical and demographical drivers of pollution.
	\item The diesel engine issue.
	\item Pollution in Barcelona: the cost in lives and healthcare assistance.
	\item Business incentive
\end{itemize}
}

\section{System borders and delimitations}
\label{s:system-borders}
\old{\begin{itemize}
	\item Long-term planning.
	\item Municipal and regional level regulation.
	\item Transport and other related policies: work environment.
\end{itemize}}

\section{Methods}
\label{s:methods}
\old{\begin{itemize}
	\item Literature review; evaluation criteria.
	\item Case study.
\end{itemize}}

\section{Evidence}
\label{s:evidence}
\old{\begin{itemize}
	\item There are potential advances in the CERO framework towards the up-scaling of the concept to broader systems and higher socio-political levels.
	\item Barcelona lacks proper transport regulation and resource administration: central state vs. regional vs. municipal.
	\item Economy is heavily based on service provision: potential for CERO framework.
\end{itemize}}

\section{Personal justification}
\label{s:personal-justification}
\old{\begin{itemize}
	\item Natural born of Barcelona.
	\item Done a course on individual choice modelling.
	\item Interested in transport regulation.
	\item Data collection and statistical evaluation, if needed, are skills gained in the computer science bachelor.
\end{itemize}}

\section{Generalisation and transferability}
\label{s:generalisation}
\old{\begin{itemize}
	\item Low generalisation, but possible insights on common regulation for other cities.
	\item The framework of the project should be transferable (backcasting and behaviour modelling).
\end{itemize}}