\chapter{Conclusion}
\label{c:conclusion}

The thesis explores a possible sustainable future of mobility and the transition pathway to it, focusing on the socio-cultural dimensions that shape and drive the way mobility is understood. Goal-driven, transition-oriented policy recommendations are provided as the main result, derived from a combined backcasting and forecasting methodology framework. The successful combination of backcasting and Causal Loop Diagrams is achieved by homogenising the outcomes of each assessment through the logic of the Multi-Level Perspective of transitions theory.

Reinforcing feedback mechanisms and a deeply embedded culture of automobility are behind the enormous inertia and resilience of the current mobility system. If a transition to a sustainable mobility future is to happen, the insights gained from this study point to a necessary shift in cultural trends. The discourses of unrestricted individual freedom, private property and materialistic cultures that legitimise automobility must be challenged. Cultural changes have to be backed by sustainable societal development trends such as fossil fuels phase-out, electrification of mobility, de-carbonisation of the power grid, high levels of sustainability concern, increases in urban density and reductions of mobility demand. Urban planning that supports public transport and hinders automobility development is of utmost importance in order to reduce the negative impacts of today's transport system.

The policy recommendations derived in this research are meant to form a coherent package. They are not understood separately and are designed with the long-term transition goal in mind. Three levels of policy recommendations are provided: niche, regime and landscape measures. The combination of all three levels should break the lock-in mechanisms of today's mobility and narrow the way for the transition to tomorrow's sustainable mobility.

While several limitations are discussed with regards to methodological choices, the overall framework is arguably the most interesting contribution to sustainability and transitions studies. The thesis proves that the Multi-Level Perspective on transitions provides with a narrative capable of integrating results from inherently different approaches to future studies. The framework is generalisable and useful for situations where a normative goal in the distant future is pursued, while accounting for the reasons behind policy resistance in the current system configuration.

Further research is needed to adapt the methodological framework to quantitative approaches. An appraisal of different scenarios (instead of only one) could also be a future research line, to assess whether or not the framework is still useful for traditional scenario exploration. Finally, participatory processes could and should be incorporated in the future, if the methodology presented in this thesis was to be used in another study.