\chapter{Introduction}
\label{c:introduction}

This thesis report is an investigation of sustainable mobility futures. It is a theoretical exploration of how policy can deliver the necessary efforts and guidelines for a successful a transition to a sustainable mobility system. \emph{What} is understood for ``sustainable'' and \emph{how} the transition is to be managed and envisioned are the main foci of the document. Special emphasis is put on the \emph{cultures} of mobility and on achieving the widest \emph{system perspective} as possible. This being said, a clarification must be made already regarding a limitation in the scope of this study: the concept of ``mobility'' that is used throughout the thesis is bound to \textbf{personal travel}, i.e., not freight transport.

Overall, the report structure adheres the IMRAD+C paradigm, with the present \nameref{c:introduction} chapter, followed by the \nameref{c:methods}, \nameref{c:results}, \nameref{c:discussion} and, finally, the \nameref{c:conclusion}. With regards to this introductory chapter, \sref{s:intro:unsustainable-mobility} presents an overview of the pressing issues of the current mobility system. A state of the art review with respect to policy approaches and design tools is given in \sref{s:intro:state-of-art}. To conclude the chapter, \sref{s:intro:aim-objectives} states the aims and objectives that motivate the rest of this research.

\section{Unsustainable mobility}
\label{s:intro:unsustainable-mobility}
Several issues of the current mobility system make it qualify as unsustainable. Direct downstream impacts, e.g. greenhouse gas emissions, are the most directly perceivable hazards, but it is their combination with global future trends that increases the significance of mobility impacts on sustainability. Issues such as population growth \parencite{un-desa2015_WorldPopulationProspects,kc2017_humancoreshared}, peak oil \parencite{kerr2011_Peakoilproduction}, expected impacts from climate change and growing economies in Asia, South-America and Africa, all highlight the acceleration and exponential expansion of the negative effects that high mobility poses to the environment, the economy and to human health and social systems.

Transport related airborne pollution is one of the main causes of respiratory diseases and associated increase in morbidity in densely populated areas~\parencite{vimercati2011_Trafficrelatedair,who2006_Airqualityguidelines}. Ambient air pollution is estimated to cause 4.4 million premature deaths around the globe~\parencite{forouzanfar2016_Globalregionalnational} and the link from air pollution to both severe health problems and high traffic volumes is well known and thoroughly researched~\parencite{who2006_Airqualityguidelines}: \ce{NO_x} emissions that lead to increases in \ce{PM_{2.5}} particulate and ozone concentrations are directly linked to diesel combustion engines, in heavy duty but also light duty vehicles \parencite{anenberg2017_Impactsmitigationexcess}. The fact that regulations and emission limits are in place within the automotive industry has not alleviated the problem, due to ever-growing automobile use and because of the industry efforts to deceive such regulations, avoiding costly research and development investments, as is the case of the recent ``dieselgate'' scandal \parencite{guardian2017_Volkswagenrevealsrecord}.

The current unsustainable mobility system not only causes respiratory health issues, but also congestion, accidents, noise pollution, infrastructure degradation and, finally, it is one of the sectors that most contribute to climate change \parencite{korzhenevych2014_UpdateHandbookExternal}. Congestion is, for example, the cause of massive costs in terms of reduced productivity, increased energy and fuel consumption, higher accident risk and its subsequent economic impacts which, for instance, are estimated at 4.2\% of Beijing's 2010 GDP\footnote{Gross Domestic Product} ~\parencite{li-zeng2012_SocialCostTraffic}. Road-related accidents alone (cars, buses and other vehicles aggregated) cause vehicle losses and damages and, most importantly, death rates that exceed 1.2 million worldwide per year or \num{28077} in the European Union (EU) in 2015 \parencite{who2017_GlobalHealthObservatory}. Safety in rail and aviation is much higher (especially per person kilometre travelled) than in roads, but they still take away 993 lives in rail-related accidents and 155 in aviation (EU data, 2015) \parencite{eurostat2017_StatisticsExplainedRailway,eurostat2017_EurostatOnlineDatabase}.

The transport sector is responsible for 27.8\% of the global final energy consumption (2014 data), with over 95\% of this energy coming from fossil sources (oil, primarily, but natural gas and coal too) \parencite{iea2017_Statisticswebportal}. \textcite{chapman2007_Transportclimatechange} already estimated that 26\% of the total world's \ce{CO_2} emissions were borne in the transport sector, of which 65\% are originated in road transport. Given the enormous pressure that climate change puts on the resilience of modern societies \parencite{ipcc2014_ClimateChange2014} and the current undertaking to tackle this global challenge --- take, for example, the recent Paris Agreement Framework Convention on Climate Change \parencite{clemencon2016_TwoSidesParis} ---, transport (mobility) is one of the sectors that must be thoroughly examined, revised and challenged to deliver urgent greenhouse gas emissions mitigation.

\subsection{Automobility at the core}
\label{ss:intro:automobility-at-core}
Automobility, as a personal mobility solution, has brought about many positive consequences, from the individuals point of view. However, its exponential growth and reliance on fossil fuels and massive infrastructures to work deem it as a global threat to the environment and, at a more local scale, to the quality of life of the very same individuals that make use of this transport mode. Given that automobility accounts for almost 36\% of the travel demand (in person kilometres per year) \parencite{vuuren2017_Energylanduse}, and that 65\% of \ce{CO_2} emissions from all travel modes are attributed to road transport \parencite{chapman2007_Transportclimatechange}, it is logical to state that automobility is, indeed, a major player in the sustainable mobility discussion. Therefore, this thesis will place a good deal of emphasis on this particular mode. The study scope encompasses the relation between this regime and public transport, or between automobility and urban planning, for example --- limiting the scope of this research to the automobile system would certainly not be sufficient to address the broad concept of sustainable mobility.

%\subsection{The need for a transition}
%\label{ss:intro:need-for-transition}
%
%\todoparagraph{Introduce the necessity of a transition (because impacts and expected growth in those due to population/affluence growth, etc. etc.)}
%
%There is therefore, at the environmental, social and economic levels, a societal need for a transition away from the current dominant regime of automobility. A new sustainable mobility paradigm is to emerge if we are truly committed to a sustainable future. How this paradigm may look like is not clear and needs further investigation from a future studies perspective.

\section{State of the art}
\label{s:intro:state-of-art}
%To address all the aforementioned issues, policy packages or simultaneous enforcing of different policies are needed, because of the complexity involved in effectively reducing transportation impacts~\parencite[ch. 3, p. 45]{garciasierra2014_Travelbehaviourenvironmental}. Regulation cannot, however, be designed without evaluating the caused impacts, at any level, in the short and long term -- too many resources and possible negative outcomes would be at stake. \textit{Policy assessment} from the systems thinking perspective is, therefore, a key issue to develop, due to the difficulty of dealing with entire systems, their internal dynamics and the emergent systemic behaviour patterns, such as feedback loops, rebound effects and hidden causalities. In this regard, the field of \textit{system dynamics} can help capture such structures and cause-effect chains \parencite{hjorth2006_Navigatingtowardssustainable}. The holistic nature of system dynamics models can help achieving an integrated assessment framework for urban mobility policies, by delivering information on a set of several indicators, at the environmental, social and economic levels.
%
%Finally, there is another perspective to be considered in the thesis. Policy development in the field of transportation has been mainly focused on two paths, according to \textcite{koehler2009_transitionsmodelsustainable}, to increase the sustainability of the system: (a) efficiency increasing measures, by means of incentives for technological enhancements (e.g. better engines or fuel mixes) and more stringent pollution limits and (b) behavioural change management, i.e., measures aimed at modal shift -- encouraging people to shift from private cars to public transport, for example. Both approaches, albeit successful to some extent, have not delivered the expected results so far, due to the high inertia and stabilization mechanisms inherent to the current dominant regime for transportation: internal combustion engines cars-based mobility \parencite{geels2012_AutomobilityTransitionSocio}. However, a third way is possible, when it comes to policy design for sustainable mobility: \textit{transition management oriented policy} (TMOP).
%
%The approach of TMOP entails adopting a longer term thinking mindset (usually one or several generations), multi-level and multi-domain thinking, maintaining support for a large set of solutions and with a focus in system \textit{innovation} alongside system \textit{improvements} \parencite{rotmans2001_Moreevolutionthan}. Moreover, flexibility in the objectives of policies is encouraged, as well as a more qualitative perspective to policy goals. All these characteristics configure a policy design mindset that could be the key to unlock a true game change in urban mobility, by drifting the focus from quantitative, efficiency measures to a transition and innovation vision, where something more than technological progress is harnessed to achieve a sustainable transport system in the future.

Historically\footnote{The historical time-frame considered in the thesis goes back to the beginning of the 20th century.}, in the ``advanced'' economies of Europe and North-America, personal mobility issues such as congestion and accessibility\footnote{Accessibility is meant as the capacity of reaching (travelling) a destination from ``any'' other given point in a territory.} have been addressed through the development of infrastructure, through increased incentives for automobility and, to some extend, travel demand management \parencite{lyons2012_VisionsFutureNeed}. Policy solutions for environmental social impacts of automobility have also been focused on technological improvement and, to some extent, modal shift\footnote{Modal shift refers to changes in the shares of travel modes (reducing car use in favour or biking, for example)} \parencite{koehler2009_transitionsmodelsustainable}.

However effective these policies were in the past --- the paradigm of ``predict-and-provide'' for infrastructure development (to address congestion) has already been dismissed in the UK, being regarded as non-efficient and even counter-productive \parencite{goodwin2012_ProvidingRoadCapacity} ---, it is clear that they are not so nowadays. Faced with pressing global trends like population growth and the fact that emergent economies in Asia and South-America are also embracing automobility as the paradigm of personal mobility, pressure keeps building on the natural and social environments. A new policy approach is needed; one that is capable of transforming the mobility system into a more sustainable one. Some efforts have been taken to fill this gap, through policy assessment tools like, for example, sustainability indicator frameworks \parencite{castillo2010_ELASTICmethodological,haghshenas2012_Urbansustainabletransportation,litman2007_DevelopingIndicatorsComprehensive,shiau2013_Developingindicatorsystem}.

Despite the best of the intentions behind them, policy assessment tools such as indicators have not performed as expected. Most indicator frameworks are not used as they should, i.e., for their instrumental and operational roles. Instead, policy makers just use them as another source of information, because they claim that ``sets of numbers'' do not convey the necessary insights for policy design or formulation \parencite{gudmundsson2013_SomeuseLittleinfluence}. One of the main drawbacks of indicators is that they are focused on \emph{impact} assessment. Policy design should not be focused on a responsive approach -- a proactive, driver-based approach is the key to a sustainable mobility transition. There is a need to model and conceptualize the ``engine'' of the mobility system and, then, find leverage points for effective policy design. Other traditional policy assessment tools, such as Integrated Assessment Models, are much broader and do evaluate the drivers of the transport sector, but fail to capture the social dimension of the system (they tend to be focused on economic variables and trends such as fuel prices) \parencite{creutzig2015_EvolvingNarrativesLow}. The hypothesis held in this study is that traditional policy assessment tools lack either the systems perspective necessary to avoid policy resistance\footnote{\emph{Policy resistance} refers to unexpected responses of a system to a certain policy measure, usually counter-acting the intended effect.} or a more normative approach to facilitate policy design.

One very important research development in the latest decades has been setting sustainable mobility \emph{visions} for the future. They form the foundation of any further policy or socio-technical development and there are examples of such, like the seminal paper by \textcite{banister2008_sustainablemobilityparadigm}, entitled ``The sustainable mobility paradigm''. However, the \emph{path} from the current situation to the desired vision of mobility remains rather unexplored. Paradigm exploration papers like Banister's provide with general policy recommendations, but do not dive deep into this realm. Additionally, they sometimes fail to account for the dynamic behaviour of the system as a whole and the mechanisms through which policy resistance is created remain rather unexplored. This is, they do not fully investigate the actual reasons why policy is sometimes ineffective, through stability and change dynamics. Moreover, these studies rarely provide with analysis of cultures or of system agents relations, thus being centred in technological, institutional and behavioural\footnote{Note that behaviour is conditioned by culture, but it is not equivalent. Behaviours can be changed within a certain practice space and still remain embedded in the same cultural framework of the previous behaviour.} aspects of mobility.

Finally, a new field of research has emerged that aims to tackle some of the shortcomings discussed in the previous paragraphs: \emph{transition studies} (or \emph{theory}). Scholars like Frank Geels, René Kemp and Jan Rotmans have spearheaded this research community, albeit some differences among their approaches: the tradition of \emph{socio-technical} transitions deals with retrospective and future studies of the changes suffered by socio-technical systems \parencite{geels2001_Technologicaltransitionsas,geels2005_DynamicsTransitionsSocio}, while the tradition of \emph{transition management} is focused on the governance of complex socio-technical systems that are meant to undergo a transition process \parencite{rotmans2001_Moreevolutionthan}. Even though this approach is explained in more detail in the \nameref{c:methods} chapter, it is worth noting that the central focus of the theory is the stability and change dynamics of the systems under study. Following \textcite{geels2001_Technologicaltransitionsas,rotmans2001_Moreevolutionthan}, transition studies investigate the co-evolution processes and multi-dimensional interactions occurring among agents (users, companies, policy makers, markets, culture, etc.) in a system. It is, therefore, a dynamics-centric, system-wide and possibly normative perspective that aims to understand how and why transitions take place in socio-technical systems.


\section{Aim and objectives}
\label{s:intro:aim-objectives}
The previous (sub)sections have covered the harmful and beneficial impacts caused by the current mobility paradigm, as well as some of the policy and scientific approaches to the solutions designed to mitigate the damaging effects of the transport system. For the sake of emphasis, automobility is, as already mentioned, the main focal point of research in terms of impact alleviation, due to its dominant position in terms of travel volume. Several gaps have also been identified, not only in terms of scientific and policy research, but also between the expected outcome of the implemented policies and their actual results (impossible demand meeting by infrastructure development, insufficient efficiency improvements, inefficient demand management schemes, etc.). This lack of major success in the scientific and political endeavours to reach a more sustainable mobility leaves a \textbf{primary research question} still open:
\blockquote{``\textit{What is} a sustainable mobility system and \textit{how} can modern societies develop the necessary adaptations to reach it?''}

%TODO rephrase the whole following paragraph
This broad question actually entails two separate but intimately related issues: (a) the definition of the \textit{concept} of sustainable mobility itself and (b) the \textit{path} from the current situation to the targeted (future) sustainable transport system. \textbf{SOMETHING IS MISSING HERE, BECAUSE IT NO LONGER BELONGS HERE!}. The second issue has been tackled in different ways because of the inherent dependency on what is understood for sustainable mobility --- i.e., the first issue --- and because different disciplines adopt different epistemic approaches \parencite{creutzig2015_EvolvingNarrativesLow}. As has been previously reviewed, a lot of attention had traditionally been brought upon technological enhancements of travel modes (especially for cars) and on infrastructure optimisations. All of these did not, however, challenge the underlying dominance of the automobile regime in the current mobility system and did not, therefore, aim for a transition to a different regime or a radical change in the way we understand personal mobility.

More recently, the concept of a sustainable mobility paradigm has evolved with an increasing understanding that automobility is one of the key causes of the negative impacts of the system. Authors and institutions have thus embraced a more proactive approach to reduce the need for automobility, especially through demand management. However, the system's inertia and resistance to change has become evident, resulting in failures to implement more drastic demand reducing policies \parencite{geels2012_AutomobilityTransitionSocio}. The hypothesis held by the scholars in the field of \textit{transition studies} (see \sref{s:intro:state-of-art}), and which this thesis shares, is that the majority of efforts made so far to reduce the ever growing impacts of mobility have failed to take into account the \textit{socio-cultural} dimension of the very concept of mobility.

Therefore, by (1) building upon the principles of sustainable mobility presented by \textcite{banister2008_sustainablemobilityparadigm}, (2) drawing on the integrated socio-technical narrative that transition studies enable and (3) acknowledging the importance of automobility as the key driver of impacts in the urban transport system, the \textbf{aim of this thesis} is:
\blockquote{To investigate how to achieve a sustainable mobility system in the future, by studying the dynamics of a socio-technical \textit{transition} away from the current dominant regime of automobility.}
This is, the focus of this study will not be on the concept of sustainable mobility, but on the \textit{transition} pathway to it and not in technological improvements or demand management techniques either, but on the socio-technical and cultural \textit{drivers} of (auto)mobility to understand what are the elements of resistance to and enablers of change in the transport system.

Regarding the target audience of the thesis, the purpose of the study is to develop insights for policy makers and other stakeholders capable of decision making in the field of mobility from the transition studies perspective. This discipline has already tackled the problem (see, for example, \textcite{geels2012_AutomobilityTransitionSocio}) but lacks, in the opinion of the author, the informative power that other policy assessment tools have. This means that, however well transition studies capture the socio-cultural aspects of the (lack of) transition in the mobility system, the nature of its epistemological background --- social sciences --- makes it more complex for key stakeholders to grasp their results. Therefore, \textbf{another important goal} of the thesis is to incorporate the transition studies perspective and discourse into more traditional and accessible policy assessment tools.

\todoparagraph{Is the study normative? In which way is it? Are there data assumptions? Is there assumptions on the future development of the world? (scenarios, etc.)}
\todoparagraph{What is the normative assumption? Possibilities: \textit{liberal} or \textit{welfarist}, according to Creutzig (2015). Alternative: \textit{modular assessment} model.}

To sum up, the following set of objectives correspond to the tasks that make up this investigation of sustainable futures of mobility:
\begin{enumerate}[leftmargin=*,label=\textbf{Obj.~\arabic*.}]
\item\label{obj:1} Analyse the transition related dynamics of the socio-cultural and technical drivers for the current automobility dominated system, in order to assess both the system's resistance to and the possibilities for change.
\item Discuss and develop a description of a desirable future mobility system. This description should be comprehensive enough to incorporate both the drivers for mobility and their downstream effects.
\item Analyse the necessary changes that separate the distant desired future from the current situation, from the point of view of transition studies.
\item Develop long term policy recommendations that are compliant with the transition based changes that have been previously identified. The appraisal of the transition dynamics in Objective 1 is incorporated to improve the insights for policy design.
\end{enumerate}

\subsection{Automobility at the core}
\label{ss:intro:automobility-at-core}
Automobility, as a personal mobility solution, has brought about many positive consequences, from the individuals point of view. However, its exponential growth and reliance on fossil fuels and massive infrastructures to work deem it as a global threat to the environment and, at a more local scale, to the quality of life of the very same individuals that make use of this transport mode. Given that automobility accounts for almost 36\% of the travel demand (in person kilometres per year) \parencite{vuuren2017_Energylanduse}, and that 65\% of \ce{CO_2} emissions from all travel modes are attributed to road transport \parencite{chapman2007_Transportclimatechange}, it is logical to state that automobility is, indeed, a major player in the sustainable mobility discussion. Therefore, this thesis will place a good deal of emphasis on this particular mode. The study scope encompasses the relation between this regime and public transport, or between automobility and urban planning, for example. Limiting the scope of this research to the automobile system would certainly not be sufficient to address the broad concept of sustainable mobility.