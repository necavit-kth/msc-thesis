\chapter{Introduction}
\label{c:introduction}

%TODO move the section to its own file and uncomment this:
%\section{Unsustainable mobility}
\label{s:intro:unsustainable-mobility}

\todo[inlne]{This section should describe the \textbf{reference mode} (Sterman, 2000): the set of graphs and descriptive data showing the development of the problem over time. \textit{Consider moving the \textit{reference mode} discussion to the Results chapter, wherever the Sys. Dyn. model is introduced.}}

\todo[inline]{Explain the situation of the transport system, with regards to sustainability issues: environmental hazards, (un)economic performance and social impacts of the system as it is. Also describe the socio-economic benefits (increased economic output due to high mobility, etc.) but highlight that mobility is envisioned as a ``right'' when it is not so, necessarily.}

\subsection{Mobility regimes and their impacts}
\label{ss:intro:mobility-regimes}

\todo[inline]{Introduce the concept of socio-technical/technological regimes. Describe, in more detail, automobility as such a regime and compare it to other mainstream transport solutions (regimes): aviation, rail, bus and cycling/walking.}
\section{Unsustainable mobility}
\label{s:intro:unsustainable-mobility}

\todo[inline]{This section should describe the \textbf{reference mode} (Sterman, 2000): the set of graphs and descriptive data showing the development of the problem over time. \textit{Consider moving the \textit{reference mode} discussion to the Results chapter, wherever the Sys. Dyn. model is introduced.}}

\todo[inline]{Explain the situation of the transport system, with regards to sustainability issues: environmental hazards, (un)economic performance and social impacts of the system as it is. Also describe the socio-economic benefits (increased economic output due to high mobility, etc.) but highlight that mobility is envisioned as a ``right'' when it is not so, necessarily.}

\todo[inline]{Frame the mobility system and its impacts and developments within the global pressing trends of increasing population, affluence and so on.}

\subsection{Mobility regimes}
\label{ss:intro:mobility-regimes}

\todo[inline]{Introduce the concept of socio-technical/technological regimes. Describe, in more detail, automobility as such a regime and compare it to other mainstream transport solutions (regimes): aviation, rail, bus and cycling/walking.}

%TODO move the section to its own file and uncomment this:
%\section{The need for a transition to sustainable mobility}
\label{s:intro:need-for-transition}


\section{The need for a transition to sustainable mobility}
\label{s:intro:need-for-transition}

\todo[inline]{Introduce the necessity of a transition (because impacts and expected growth in those due to population/affluence growth, etc. etc.)}

There is therefore, at the environmental, social and economic levels, a societal need for a transition away from the current dominant regime of automobility. A new sustainable mobility paradigm is to emerge if we are truly committed to a sustainable future. How this paradigm may look like is not clear and needs further investigation from a future studies perspective.

%TODO move the section to its own file and uncomment this:
%\section{State of the art}
\label{s:intro:state-of-art}

\todo[inline]{Explain solutions or, rather, approaches (both in scientific literature and management/policy-making settings) to solving the unsustainable situation of the transport system.}

\subsection{Sustainable mobility indicators}
\label{ss:intro:sustainable-mobility-indicators}

\todo[inline]{According to [SOME PAPER], most indicator frameworks are not used as they should: for their instrumental/operational role. Instead, policy makers just use them as another source of knowledge/information, because they claim that ``sets of numbers'' do not convey the necessary insights for policy design or formulation. \textbf{RESEARCH GAP:} Indicators are focused on impact assessment, while policy design should not be focused on a responsive approach -- a proactive, driver-based approach is the key to a sustainable mobility transition. There is a need to model and conceptualize the ``engine'' of the mobility system and, then, find leverage points for effective policy design.}

\subsection{Sustainable socio-technical transitions}
\label{ss:intro:sustainable-transitions}

\todo[inline]{Provide an overview of how do transition studies deal with the issue of unsustainable mobility and, in particular, automobility. What are regimes? What are niches? What are the main approaches? (Section 3.2 in Geels, Kemp et al., 2012)}

\todo[inline]{\textbf{RESEARCH GAP:} To which extent do transition studies address the long term planning of mobility? Transition management does, but might be combined with other planning tools, such as modelling and forecasting and backcasting for long term visions.}
\section{State of the art}
\label{s:intro:state-of-art}

\todo[inline]{Explain solutions or, rather, approaches (both in scientific literature and management/policy-making settings) to solving the unsustainable situation of the transport system.}

\subsection{Traditional policy practices}
\label{ss:intro:traditional-policy-practices}

\todo[inline]{Talk about the historic development of automobility and planning/policy schemes used throughout the past century (\textit{predict-and-provide} and \textit{demand management}).}

\subsection{Policy assessment tools}
\label{ss:intro:policy-assessment-tools}
\todo[inline]{Talk about: (a) indicators, (b) Integrated Assessment Models, (c) travel demand prognosis tools, (c) etc.?}

\subsubsection{Sustainable mobility indicators}
\label{sss:intro:sustainable-mobility-indicators}

\todo[inline]{According to [SOME PAPER], most indicator frameworks are not used as they should: for their instrumental/operational role. Instead, policy makers just use them as another source of knowledge/information, because they claim that ``sets of numbers'' do not convey the necessary insights for policy design or formulation. \textbf{RESEARCH GAP:} Indicators are focused on impact assessment, while policy design should not be focused on a responsive approach -- a proactive, driver-based approach is the key to a sustainable mobility transition. There is a need to model and conceptualize the ``engine'' of the mobility system and, then, find leverage points for effective policy design.}

\subsection{Sustainable socio-technical transitions}
\label{ss:intro:sustainable-transitions}

\todo[inline]{Provide an overview of how do transition studies deal with the issue of unsustainable mobility and, in particular, automobility. What are regimes? What are niches? What are the main approaches? (Section 3.2 in Geels, Kemp et al., 2012) Highlight that these studies actually do stress the cultural component of automobility. Without acknowledging it, there is no way that a transition is successful and, even less, that it can be managed/accelerated/supported by policy means.}

\todo[inline]{\textbf{RESEARCH GAP:} To which extent do transition studies address the long term planning of mobility? Transition management does, but might be combined with other planning tools, such as modelling and forecasting and backcasting for long term visions.}

%TODO move the section to its own file and uncomment this:
%\section{Aim and objectives}
\label{s:intro:aim-objectives}
\section{Aim and objectives}
\label{s:intro:aim-objectives}

The previous (sub)sections have covered the harmful and beneficial impacts caused by the current mobility paradigm, as well as some of the policy and scientific approaches to the assessment and design of solutions to the damaging effects of the transport system. However, several gaps have also been identified, not only in terms of scientific and political research, but also between the expected outcome of the implemented policies and their actual results. The causes of such inefficacies remain subject to an intense debate within the diverse research communities that try to tackle the challenge of sustainable mobility\todo{cite that paper talking about IAMs, public health and transport specialists (bringing together their narratives)}. The hypothesis held by this author is that the majority of efforts made so far to reduce the ever growing impacts of mobility have failed to take into account the socio-cultural dimension of the very concept of mobility.

Because of the lack of major success in the scientific and political endeavours to reach a more sustainable mobility, a primary or background research question remains open: \textit{What is a sustainable mobility system and how can modern societies reach a sufficient level of sustainability in such a complex system?}

\todo{Re-formulate or move some bits to some other places}The primary research question is: \textit{What are the alternatives for a transition away from the dominant automobility regime?} In order to answer this broad and open question, new methodological frameworks to investigate the possible future development paths of personal mobility are needed. This is so because information tools for policy makers such as indicator frameworks have failed to enable long-term and comprehensive system innovation in the transport sector, for example. A broader perspective is required to study the issue, since the transition entails not only technological or environmental aspects, but cultural and socio-political ones too. Transition studies provide, on one hand, with the appropriate framework to include said cultural and social components, but fail, on the other hand, to inform policy makers with the type of information that they need in order to judge the decisions to be made. This is, transition studies capture socio-cultural aspects of mobility well, but they are not so well suited to convey the information and knowledge they produce to the key stakeholders that play a part in the transition to a more sustainable mobility.

The aim is to ``expand'' more traditional policy assessment tools with the perspective of transition studies to enable a better understanding of the problems of the current mobility system and its potential solutions. Therefore, the idea is to frame these more conventional tools, which convey information in an easier way, within the conceptual framework of transition studies. This way, the problems can be reformulated and ``attacked'' from a new perspective -- one in which the deep cultural roots of the current dominant regime can be analysed and challenged to find alternative paths of development in the future.

The aim is to investigate how to achieve a future sustainable mobility system, by studying the dynamics of a socio-technical transition away from the current dominant regime of automobility.

\todo{Move this to Methods}The combination of assessment and planning tools, embedded in the discourse of transition studies, is used to discuss possible global patterns of development, by means of a dual backcasting and forecasting approach: scenario backcasting on one hand and system dynamics modelling on the other.

\todo{Move this to results/discussion}Change is to be brought upon by two forces: a different conceptualisation of (personal) mobility and the changes in infrastructure, transport provision and life style that would accompany such change.