\chapter{Introduction}
\label{c:introduction}

\todoparagraph{Specify the scope!! Which is, in principle, passenger travel. Not freight. Not shipping. That is beyond the scope of the thesis. It is an entirely different system. It has more to do with long distances and global logistic networks than with urban and inter-urban travel. It has nothing to do with commuting nor with the associated problems of congestion, accidents and high pollution levels of the automobile mobility world.}

%TODO move the section to its own file
\section{Unsustainable mobility}
\label{s:intro:unsustainable-mobility}

\todoparagraph{This section should describe the \textbf{reference mode} (Sterman, 2000): the set of graphs and descriptive data showing the development of the problem over time. \textit{Consider moving the \textit{reference mode} discussion to the Results chapter, wherever the Sys. Dyn. model is introduced.}}

\todoparagraph{Explain the situation of the transport system, with regards to sustainability issues: environmental hazards, (un)economic performance and social impacts of the system as it is. Also describe the socio-economic benefits (increased economic output due to high mobility, etc.) but highlight that mobility is envisioned as a ``right'' when it is not so, necessarily.}

\todoparagraph{Frame the mobility system and its impacts and developments within the global pressing trends of increasing population, affluence and so on.}

\subsection{Mobility regimes}
\label{ss:intro:mobility-regimes}

\todoparagraph{Introduce the concept of socio-technical/technological regimes. Describe, in more detail, automobility as such a regime and compare it to other mainstream transport solutions (regimes): aviation, rail, bus and cycling/walking.}

\todoparagraph{IMPORTANT: in order for the reader to understand the SSP1-MOB narrative down the report, explain the concept of automobility that is meant throughout the thesis. This is, automobility is not only cars, but any kind of privately-owned, motorised form of mobility (two-wheelers such as motorbikes also included!).}

%TODO move the section to its own file
\section{The need for a transition}
\label{s:intro:need-for-transition}

\todoparagraph{Introduce the necessity of a transition (because impacts and expected growth in those due to population/affluence growth, etc. etc.)}

There is therefore, at the environmental, social and economic levels, a societal need for a transition away from the current dominant regime of automobility. A new sustainable mobility paradigm is to emerge if we are truly committed to a sustainable future. How this paradigm may look like is not clear and needs further investigation from a future studies perspective.

%TODO move the section to its own file
\section{State of the art}
\label{s:intro:state-of-art}

\todoparagraph{Explain solutions or, rather, approaches (both in scientific literature and management/policy-making settings) to solving the unsustainable situation of the transport system.}

\subsection{Traditional policy practices}
\label{ss:intro:traditional-policy-practices}

\todoparagraph{Talk about the historic development of automobility and planning/policy schemes used throughout the past century (\textit{predict-and-provide} and \textit{demand management}).}

\subsection{Policy assessment tools}
\label{ss:intro:policy-assessment-tools}
\todoparagraph{Talk about: (a) indicators, (b) Integrated Assessment Models, (c) travel demand prognosis tools, (c) etc.?}

\subsubsection{Sustainable mobility indicators}
\label{sss:intro:sustainable-mobility-indicators}

\todoparagraph{According to \textcite{gudmundsson2013_SomeuseLittleinfluence}, most indicator frameworks are not used as they should: for their instrumental/operational role. Instead, policy makers just use them as another source of knowledge/information, because they claim that ``sets of numbers'' do not convey the necessary insights for policy design or formulation. \textbf{RESEARCH GAP:} Indicators are focused on impact assessment, while policy design should not be focused on a responsive approach -- a proactive, driver-based approach is the key to a sustainable mobility transition. There is a need to model and conceptualize the ``engine'' of the mobility system and, then, find leverage points for effective policy design.}

\subsection{Sustainable mobility paradigms}
\label{ss:intro:sustainable-mobility}

\todoparagraph{Describe the efforts so far to define the concept/paradigm of sustainable mobility. Especially, focus on the insights derived from \textcite{banister2008_sustainablemobilityparadigm} and similar papers (is there anything in the Geels 2012 book?)}

\todoparagraph{Concerning the ideas in \textcite{banister2008_sustainablemobilityparadigm}, talk about the 4 key elements to develop (in policy-ish) and the highlight of stakeholder participation as the foundation. Regarding this, there are some pairs of opposing forces (personal utility vs social welfare), (active involvement vs passive persuasion) that deserve to be mentioned.}

\todoparagraph{\textcite{banister2008_sustainablemobilityparadigm} does stress that the behavioural aspects of mobility deserve much further attention, and presents the notion of public acceptability as a key to effective policy introduction, but still misses (not too much, admittedly) the point of \textit{culture} and intrinsic moral/personal values in the automobility conceptualisation of our society. This deserves as well a lot of attention.}

\subsection{Socio-technical transitions}
\label{ss:intro:transitions-theory}

\todoparagraph{Provide an overview of how do transition studies deal with the issue of unsustainable mobility and, in particular, automobility. What are regimes? What are niches? What are the main approaches? (Section 3.2 in Geels, Kemp et al., 2012) Highlight that these studies actually do stress the cultural component of automobility. Without acknowledging it, there is no way that a transition is successful and, even less, that it can be managed/accelerated/supported by policy means.}

\todoparagraph{\textbf{RESEARCH GAP:} To which extent do transition studies address the long term planning of mobility? Transition management does, but might be combined with other planning tools, such as modelling and forecasting and backcasting for long term visions.}

%TODO move the section to its own file
\section{Aim and objectives}
\label{s:intro:aim-objectives}

The previous (sub)sections have covered the harmful and beneficial impacts caused by the current mobility paradigm, as well as some of the policy and scientific approaches to the solutions designed to mitigate the damaging effects of the transport system. For the sake of emphasis, automobility is, as already mentioned, the main focal point of research in terms of impact alleviation, due to its dominant position in terms of travel volume. Several gaps have also been identified, not only in terms of scientific and policy research, but also between the expected outcome of the implemented policies and their actual results (\todonote{All of these should be mentioned in the previous sections!}impossible demand meeting by infrastructure development, insufficient efficiency improvements, inefficient demand management schemes, etc.). This lack of major success in the scientific and political endeavours to reach a more sustainable mobility leaves a \textbf{primary research question} still open:
\blockquote{``\textit{What is} a sustainable mobility system and \textit{how} can modern societies develop the necessary adaptations to reach it?''}

This broad question actually entails two separate but intimately related issues: (a) the definition of the \textit{concept} of sustainable mobility itself and (b) the \textit{path} from the current situation to the targeted (future) sustainable transport system. Arguably, it is the first of these issues that has been discussed the most in the literature --- it is the foundation of any further policy or socio-technical development ---, such as the \todonote{Check whether or not ``seminal'' makes any sense here! What you want to convey is that it has over a thousand citations.}seminal paper by \textcite{banister2008_sustainablemobilityparadigm}. The second issue has been tackled in different ways because of the inherent dependency on what is understood for sustainable mobility --- i.e., the first issue --- and because different disciplines adopt different epistemic approaches \parencite{creutzig2015_EvolvingNarrativesLow}. As has been previously reviewed, a lot of attention had traditionally been brought upon technological enhancements of travel modes (especially for cars) and on infrastructure optimisations. All of these did not, however, challenge the underlying dominance of the automobile regime in the current mobility system and did not, therefore, aim for a transition to a different regime or a radical change in the way we understand personal mobility.

More recently, the concept of a sustainable mobility paradigm has evolved with an increasing understanding that automobility is one of the key causes of the negative impacts of the system. Authors and institutions have thus embraced a more proactive approach to reduce the need for automobility, especially through demand management. However, the system's inertia and resistance to change has become evident, resulting in failures to implement more drastic demand reducing policies \parencite{geels2012_AutomobilityTransitionSocio}. The hypothesis held by the scholars in the field of \textit{transition studies} (see \ssref{ss:intro:sustainable-transitions}), and which this thesis shares, is that the majority of efforts made so far to reduce the ever growing impacts of mobility have failed to take into account the \textit{socio-cultural} dimension of the very concept of mobility.

Therefore, by (1) building upon the principles of sustainable mobility presented by \textcite{banister2008_sustainablemobilityparadigm}, (2) drawing on the integrated socio-technical narrative that transition studies enable and (3) acknowledging the importance of automobility as the key driver of impacts in the urban transport system, the \textbf{aim of this thesis} is:
\blockquote{To investigate how to achieve a sustainable mobility system in the future, by studying the dynamics of a socio-technical \textit{transition} away from the current dominant regime of automobility.}
This is, the focus of this study will not be on the concept of sustainable mobility, but on the \textit{transition} pathway to it and not in technological improvements or demand management techniques either, but on the socio-technical and cultural \textit{drivers} of (auto)mobility to understand what are the elements of resistance to and enablers of change in the transport system.

Regarding the target audience of the thesis, the purpose of the study is to develop insights for policy makers and other stakeholders capable of decision making in the field of mobility from the transition studies perspective. This discipline has already tackled the problem (see, for example, \textcite{geels2012_AutomobilityTransitionSocio}) but lacks, in the opinion of the author, the informative power that other policy assessment tools have. This means that, however well transition studies capture the socio-cultural aspects of the (lack of) transition in the mobility system, the nature of its epistemological background --- social sciences --- makes it more complex for key stakeholders to grasp their results. Therefore, \textbf{another important goal} of the thesis is to incorporate the transition studies perspective and discourse into more traditional and accessible policy assessment tools.

To sum up, the following set of objectives correspond to the tasks that make up this investigation of sustainable futures of mobility:
\begin{enumerate}[leftmargin=*,label=\textbf{Obj.~\arabic*.}]
\item\label{obj:1} Analyse the transition related dynamics of the socio-cultural and technical drivers for the current automobility dominated system, in order to assess both the system's resistance to and the possibilities for change.
\item Discuss and develop a description of a desirable future mobility system. This description should be comprehensive enough to incorporate both the drivers for mobility and their downstream effects.
\item Analyse the necessary changes that separate the distant desired future from the current situation, from the point of view of transition studies.
\item Develop long term policy recommendations that are compliant with the transition based changes that have been previously identified. The appraisal of the transition dynamics in Objective 1 is incorporated to improve the insights for policy design.
\end{enumerate}