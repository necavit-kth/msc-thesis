\chapter{Methods}
\label{c:methods}

\section{Normative assumption}
\label{s:methods:normative-assumption}

Following \todo[inline]{cite: Creutzig, 2015}, the normative assumption for the current study is made explicit, in order to clearly state the position and point of view of the author, as well as to ease the comparison to other studies and the reconciliation of the results with other approaches.

\todo[inline]{What is the normative assumption? Possibilities: \textit{liberal} or \textit{welfarist}, according to Creutzig (2015). Alternative: \textit{modular assessment} model.}