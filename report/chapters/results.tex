\chapter{Results}
\label{c:results}

The \gls{IPCC} has recently developed several scenarios that are the basis of their integrated assessments\todo{citation}. These scenarios are called \glspl{SSP} and are defined on the basis of qualitative narratives that contain all the necessary information on global trends to enable a further quantification step, using \glspl{IAM}, such as the IMAGE model. Given that \glspl{SSP} are not design solely for the purpose of climate change studies, but are rather a description of world futures, they can be used in other disciplines and, particularly, in any kind of sustainability studies \todo[inline]{cite: SSPs paper}. Such an extension of one of the scenarios is provided in \autoref{s:ssp1-mob}, regarding developmental paths of mobility.

\section{SSP1-MOB: a mobility extension of SSP1}
\label{s:ssp1-mob}

Among the \glspl{SSP} scenarios, SSP1 ``Sustainability -- Taking the green road'' is the one that implies a lower level of both adaptation and mitigation challenges, with respect to climate change. Moreover, it is the one that is more aligned with the concept of \gls{SD}, due to its relatively high performance in all three pillars of sustainability: environmental conservation, social and economic sustainability (at least, economic \textit{growth} per capita). Therefore, it is the selected \gls{SSP} to extend to cover the mobility sector, in order to perform the backcasting process that will be used to identify the necessary changes and development goals to reach a sustainable transport system in the future.

The extension of the scenario is called \gls{SSP1-MOB} and is also developed through a qualitative narrative, in which a \textit{vision} of a future sustainable mobility system is outlined. The features and trends of said vision are summarised in \autoref{t:ssp1-mob-narrative-vars}. The following is the narrated version of the \gls{SSP1-MOB} scenario, describing the global situation of the mobility system in the year 2100:

\blockquote{Driven by an increasing awareness level of environmental and socio-economic impacts of the transportation system, the world has adopted a series of measures and changes to reduce the aforementioned impacts.

Demand has grown, but measures to make it smaller

Private mobility (auto-mobility) is still a significant mode in terms of total travel demand, but the regime has changed: electric cars are the dominant technology used for short to medium ranged trips, such as commuting, while hydrogen-fuelled vehicles take the lead for longer trip distances. While accessibility is kept at a high level due to the possibility to use this private transport mode, car sharing is common and most urban communities benefit from reduced fleets thanks to carpooling, which is also common and well accepted\todo{Justify this. Perhaps: ``the reduced need for long trips, a higher concern for material depletion (especially metals) and increased costs both in the ownership and usage of private vehicles''? Also: IT spread for car sharing and pooling}.

Most of the travel demand is, however, supplied through public transport, be it in the form of aviation, passenger rail or buses. An increased and continuous heavy investment in public transport infrastructure (an extensive railway network, for example) has enabled fast, secure and low-carbon transport options for the majority of the population, which lives in more concentrated urban areas. High speed trains cover the demand for regional and national trips, while regular trains are used mainly by commuters and urban travellers. Efficient, bio-fuelled buses are used for short trips within cities or to connect less accessible areas.

\todo{Is this too ``micro''? The level of detail is way higher than in the other paragraphs!}Walking and cycling are increasingly adopted by many to cover very short or inner-urban trips, especially amongst the youngest. Cycling lanes are an integral part of every urban area road network and public cycling facilities, such as parking stations, are commonplace. Traffic regulation is changed to prioritise and ensure the safety of both cyclists and pedestrians. Ample footpaths (sidewalks) provide not only the space for walking but also a more ``livable'' urban environment.

}


\begin{table}
\caption{Qualitative variables and trends underlying to the \gls{SSP1-MOB} narrative.}
\begin{tabular}{l}
Foo
\end{tabular}
\label{t:ssp1-mob-narrative-vars}
\end{table}