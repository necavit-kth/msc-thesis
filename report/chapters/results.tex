\chapter{Results}
\label{c:results}

The \gls{IPCC} has recently developed several scenarios that are the basis of their integrated assessments\todo{citation}. These scenarios are called \glspl{SSP} and are defined on the basis of qualitative narratives that contain all the necessary information on global trends to enable a further quantification step, using \glspl{IAM}, such as the IMAGE model. Given that \glspl{SSP} are not design solely for the purpose of climate change studies, but are rather a description of world futures, they can be used in other disciplines and, particularly, in any kind of sustainability studies \todo[inline]{cite: SSPs paper}. Such an extension of one of the scenarios is provided in \autoref{s:ssp1-mob}, regarding developmental paths of mobility.

\section{SSP1-MOB: a mobility extension of SSP1}
\label{s:results:ssp1-mob}

Among the \glspl{SSP} scenarios, SSP1 ``Sustainability -- Taking the green road'' is the one that implies a lower level of both adaptation and mitigation challenges, with respect to climate change. Moreover, it is the one that is more aligned with the concept of \gls{SD}, due to its relatively high performance in all three pillars of sustainability: environmental conservation, social and economic sustainability (at least, economic \textit{growth} per capita). Therefore, it is the selected \gls{SSP} to extend to cover the mobility sector, in order to perform the backcasting process that will be used to identify the necessary changes and development goals to reach a sustainable transport system in the future.

The extension of the scenario is called \gls{SSP1-MOB} and is also developed through a qualitative narrative, in which a \textit{vision} of a future sustainable mobility system is outlined. The features and trends of said vision are summarised in \autoref{t:ssp1-mob-narrative-vars}. The following is the narrated version of the \gls{SSP1-MOB} scenario, describing the global situation of the mobility system in the year 2100:

\blockquote{Driven by an increasing awareness level of environmental and socio-economic impacts of the transportation system, the world has adopted a series of changes to reduce those. Vehicles have become more efficient, liquid hydrocarbon fuels are less carbon intensive and renewable (based on biofuels), but there has also been a shift in travel modes and total demand per capita (travelled kilometres per person and year) has been reduced.

Private mobility (auto-mobility) is still a significant mode in terms of total travel demand: electric cars are the main technological alternative used for short to medium ranged trips, such as commuting, while hydrogen-fuelled vehicles take the lead for longer trip distances. While accessibility is kept at a high level due to the possibility to use this private transport mode, car sharing is common and most urban communities benefit from reduced fleets thanks to carpooling, which is also commonly available and well accepted by the public\todo{Justify this. Perhaps: ``the reduced need for long trips, a higher concern for material depletion (especially metals) and increased costs both in the ownership and usage of private vehicles''? Also: IT spread for car sharing and pooling}.

Most of the travel demand is, however, supplied through \gls{PT}, be it in the form of aviation, passenger rail or buses. An increased and continuous heavy investment in \gls{PT} infrastructure (an extensive railway network, for example) has enabled fast, secure and low-carbon transport options for the majority of the population, which lives in more concentrated urban areas. High speed trains cover the demand for regional and national trips, while regular trains are used mainly by commuters and urban travellers. Efficient, bio-fuelled buses are used for short trips within cities or to connect less accessible areas.

\todo{Is this too ``micro''? The level of detail is way higher than in the other paragraphs!}Walking and cycling are increasingly adopted by many to cover very short or inner-urban trips, especially amongst the youngest. Cycling lanes are an integral part of every urban area road network and public cycling facilities, such as parking stations, are commonplace. Traffic regulation is changed to prioritise and ensure the safety of both cyclists and pedestrians. Ample footpaths (sidewalks) provide not only the space for walking but also a more ``livable'' urban environment.
}

\todo{Fix the caption of this table}
{\scriptsize
\begin{longtable}{p{3cm}p{3.5cm}p{8cm}}
\toprule
Category & Variable & Trend \\ \midrule
\multicolumn{3}{l}{\textbf{SSP1 -- Source: O'Neill et al., 2017}}\\
\textit{Demographics} & Population growth & Relatively low\\
\textit{} & Fertility rate & Low in currently high- and low-fertility countries; medium in rich OECD countries.\\
\textit{} & Mortality & Low\\
\textit{} & Migration & Medium\\
\textit{} & Urbanization level & High\\
\textit{} & Urbanization type & Well managed\\
\textit{Human development} & Education & High\\
\textit{} & Health investments & High\\
\textit{} & Access to health facilities, water, sanitation & High\\
\textit{} & Gender equality & High\\
\textit{} & Equity & High\\
\textit{} & Social cohesion & High\\
\textit{} & Societal participation & High\\
\textit{Economy \& lifestyle} & Growth (per capita) & High in LICs and MICs, medium in HICs\\
\textit{} & Inequality & Reduced across and within countries\\
\textit{} & International trade & Moderate\\
\textit{} & Globalization & Connected markets, regional production\\
\textit{} & Consumption and diet & Low growth in material consumption, low-meat diets, first in HICs\\
\textit{Policies \& institutions} & International cooperation & Effective\\
\textit{} & Environmental policy & Improved management of local and global issues; tighter regulation of pollutants\\
\textit{} & Policy orientation & Toward sustainable development\\
\textit{} & Institutions & Effective at national and international levels\\
\textit{Technology} & Development & Rapid\\
\textit{} & Transfer & Rapid\\
\textit{} & Energy tech. change & Directed away from fossil fuels, toward efficiency and renewables\\
\textit{} & Carbon intensity & Low\\
\textit{} & Energy intensity & Low\\
\textit{Environment \& natural resources} & Fossil constraints & Preferences shift away from fossil fuels\\
\textit{} & Environment & Improving conditions over time\\
\textit{} & Land use & Strong regulations to avoid environmental tradeoffs\\
\textit{} & Agriculture & Improvements in agricultural productivity; rapid diffusion of best practices\\ && \\
\multicolumn{3}{l}{\textbf{SSP1 (implementation) -- Source: van Vuuren et al., 2017}}\\
\textit{Energy demand} & Transport & Lower share of income spent on transport leading to less kms travelled. More travel time (0.5 min/day increase each year) resulting in less shift to faster mode. Preference for public transport, car sharing, and faster increase in efficiency (10\% in 2100).\\
\textit{Energy supply and conversion} & Fossil fuels & Global trade of fuels; and median technology development for fossil fuel extraction technologies.\\
\textit{} & Bio-energy & Traditional bio-fuels mostly phased out around 2030; bio-fuels in transport taxed for possible biodiversity damage; less potential based on nature reserves but increased from abandoned lands; high yields; improved efficiencies and costs of biofuel production technologies; residues based on Daioglou et al. (2016).\\
\textit{Other (in-text)} & Sustainable development agenda & In order to pursue an ambitious agenda, the main requirement is the further growth of societal support for such a strategy combined with an actual change in investment patterns (Ocampo, 2011).\\
\textit{} & Socio-technical transition & The breaking off the current trends can be achieved through the up-scaling of niches to a mainstream (regime) level (Geels, 2012). Elements of the transition include the adoption of green growth concepts, the recent approval of the SDGs but also the rapid decline in costs of key technologies such as PV and electric batteries (IRENA, 2014; Nykvist and Nilsson, 2015).\\
\textit{} & Risks of the SSP1 world & a) non-performance of the technology; b) rebound impact of efficiency; c) possible tensions associated with free-rider behaviour and d) a potential push-back from actors whose interests are not ensured in this storyline.\\
\textit{} & Transport energy & Up to 2050, alternative fuels rapidly gain market shares, but oil remains important. In 2100, there is a dominant position of electric and hydrogen-fuelled drive-trains in road transport; biofuels become the most important for aviation and trucks.\\
\textit{} & Electricity use & Rapid growth\\
\textit{} & Electricity production & 65\% renewables by 2100.\\ \bottomrule
\caption{Qualitative variables and trends underlying to the \gls{SSP1-MOB} narrative.}
\label{t:ssp1-mob-narrative-vars}
\end{longtable}
}