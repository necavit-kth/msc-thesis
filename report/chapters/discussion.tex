\chapter{Discussion}
\label{c:discussion}

\todoparagraph{Introduce the discussion chapter (structure)}


\todoparagraph{In general, the thesis ``outcomes'' are biased, personal and not too generalizable. But! The combination of policy-making tools through the ``discoursive''/``narrative'' power of transition studies is a really interesting result. Even though more work needs to be devoted to this, it clearly suggests that by using the terminology of transition theories, we can bridge the gap between seeminlgy different approaches and tools. This can be very useful in the context of policy-making and (future(s)) sustainability studies, since it promises to deliver a more coherent and integrated result than other approaches.}

\section[Limits of the SSP1-MOB vision]{Limits of the SSP1-MOB vision}
\label{s:discussion:limitations-ssp1-mob}
\input{sections/discussion/limitations-ssp1-mob}

\todoparagraph{
-- normative assumptions\\
-- SSP1 vs SSP\{2,3,4,5\} $\rightarrow$ what would happen?\\
-- \textit{realistic} vs \textit{idealistic} approaches: ``it's the backcasting, stupid!''; i.e., backcasting from a vision cannot pretend to be realistic. This is in clear contrast to the common setting for scientific research, which is based in forecasting methodologies.\\
}

\section[Missing features in the AUTOLOCK model]{Missing features in the AUTOLOCK model}
\label{s:discussion:missing-features-autolock}
\input{sections/discussion/missing-features-autolock}

\todoparagraph{
-- lack of ``fuels'' perspective and other technological aspects\\
-- lack of transition-management features (really? or is it done on another level, above the model?)\\
-- lack of other perspectives, due to (a) lack of time and (b) the fact that a ``complete'' model is out of scope and reach for this thesis.
}

\section[Quantitative analysis]{Quantitative analysis}
\label{s:discussion:quantitative-analysis}
\input{sections/discussion/quantitative-analysis}

\todoparagraph{
-- lack of quantitative figures for both the narrative/backcasting portion of the results and the CLD\\
-- lack of simulation analysis to assess the extent and pace of change in the dynamic feedback structure of the system
}