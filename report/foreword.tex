%%%% Foreword page %%%%
\cleardoublepage % Clear the previous page
\phantomsection % Create Hyperref anchor
\addcontentsline{toc}{chapter}{Foreword} % Add the abstract to the Table of Contents

\chapter*{Foreword} % Foreword section, unnumbered ("*")
This thesis, far from giving an accurate or specific set of sustainable mobility policy recommendations, is indeed more of a mental exercise. It is, above all, an exploration of new ways of doing sustainability research. Even though the original purpose is to analyse a transition towards sustainable future (personal) mobility, it is clear that no master thesis can comprehensibly tackle the issue. Not only due to time and resource constraints, but because such a transition requires highly participatory democratic processes to become a reality. This thesis does not, therefore, intend to provide with the ultimate truth with respect to sustainable mobility, but to raise to the challenge of coming up with new policy and research ways.

Some may find the title of this thesis misleading. There is, indeed only \emph{one} possible future of sustainable mobility assessed here. However, the study is embedded in the field of \emph{futures} research. Above all, the title is meant to convey that, despite the efforts in forecasting and research, the possibilities are many: a whole array of futures is ahead. It is in the hands of researchers, policy makers, stakeholders and everyone to follow one or another vision of the future and gain it for themselves.

The thesis also embraces a perspective on sustainability that is often avoided or regarded as biased: normative research. This is, the thesis diverges from descriptive studies and delves into \emph{how} sustainable mobility can be achieved. Even though such research entails challenges to fundamental values of society and societal organisation patterns, one should not hurry to regard it as politically-biased. Instead, the question that must be raised is whether or not these challenges to values and lifestyles are well-founded and are effectively necessary to reach a sustainable form of existence. This is, can we truly achieve sustained human life on Earth with our current way of living? And with our current mental frames?

Finally, the thesis intendedly avoids using too many numeric figures in the assessments contained. This is for two reasons: (1) the inclusion of numbers for macroscopic trends, such as city density, in the long-term scope of the thesis would introduce a source of uncertainty and (2) the will to emphasise that qualitative research can be as good as quantitative science to answer sustainability questions. Sometimes, change for good is not about a particular figure, but a particular course of action, a final vision---an idea.

\cleardoublepage % Clear the page for other sections (front matter or main matter)
