\clearpage
\chapter{IPCC SSP1 narrative}
\label{a:ssp1-original-narrative}

The following corresponds to the original SSP1 scenario narrative. Reprinted from Global Environmental Change, \textbf{42}, Brian C. O’Neill, Elmar Kriegler, Kristie L. Ebi, Eric Kemp-Benedict, Keywan Riahi, Dale S. Rothman, Bas J. van Ruijven, Detlef P. van Vuuren, Joern Birkmann, Kasper Kok, Marc
Levy, William Solecki, \textit{The roads ahead: Narratives for shared socioeconomic pathways describing world futures in the 21st century}, 169-180, Copyright (2017), with permission from Elsevier. License number 4104690633321.
%
\blockquote{\small
The world shifts gradually, but pervasively, toward a more sustainable path, emphasizing more inclusive development that respects perceived environmental boundaries. Increasing evidence of and accounting for the social, cultural, and economic costs of environmental degradation and inequality drive this shift. Management of the global commons slowly improves, facilitated by increasingly effective and persistent cooperation and collaboration of local, national, and international organizations and institutions, the private sector, and civil society. Educational and health investments accelerate the demographic transition, leading to a relatively low population. Beginning with current high-income countries, the emphasis on economic growth shifts toward a broader emphasis on human well-being, even at the expense of somewhat slower economic growth over the longer term. Driven by an increasing commitment to achieving development goals, inequality is reduced both across and within countries. Investment in environmental technology and changes in tax structures lead to improved resource efficiency, reducing overall energy and resource use and improving environmental conditions over the longer term. Increased investment, financial incentives and changing perceptions make renewable energy more attractive. Consumption is oriented toward low material growth and lower resource and energy intensity. The combination of directed development of environmentally friendly technologies, a favorable outlook for renewable energy, institutions that can facilitate international cooperation, and relatively low energy demand results in relatively low challenges to mitigation. At the same time, the improvements in human well-being, along with strong and flexible global, regional, and national institutions imply low challenges to adaptation.
}