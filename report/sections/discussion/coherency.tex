The thesis is, despite the aforementioned limitations, aligned with the current body of literature on sustainable mobility futures. For example, \textcite{banister2008_sustainablemobilityparadigm} describes four main categories of action that must be pursued to reach the sustainable mobility paradigm he argues for:
\begin{enumerate}
\item Reduce the need for travel through substitution.
\item Encourage modal shift through transport policy measures.
\item Reduce trip lengths through sustainable spatial planning.
\item Encourage higher efficiency in transport through vehicle and fuel technology innovation.
\end{enumerate}
This thesis, on the other side, advocates for very similar results. The backcasted changes are also categorised in four main fronts of action: (1) cultural changes to reduce the overall mobility demand, (2) land-use changes to reduce trip lengths and demand, (3) encouraging modal shift through demand management and intermodal travel and (4) increase vehicle and fuel efficiencies. This similarity is no coincidence. Both \textcite{banister2008_sustainablemobilityparadigm} and this thesis are built on the realisation that the current and historic paradigm of mobility is obsolete from a sustainability point of view and both studies focus on the transformation (or transition) of the system into a new paradigm.



\todoparagraph{In general, the thesis ``outcomes'' are ``biased'', personal and not too generalizable. But! The combination of policy-making tools through the ``discoursive''/``narrative'' power of transition studies is a really interesting result. Even though more work needs to be devoted to this, it clearly suggests that by using the terminology of transition theories, we can bridge the gap between seeminlgy different approaches and tools. This can be very useful in the context of policy-making and (future(s)) sustainability studies, since it promises to deliver a more coherent and integrated result than other approaches.}