Regarding the methods used throughout the thesis, a series of pitfalls have been identified. First and foremost, the development of the SSP1-MOB qualitative narrative, as well as the subsequent backcasting process, could and probably should have been performed on the basis of a participatory process. Even though the normative approach of those sections has a higher saliency for policy and decision makers, the legitimacy of the results might have been compromised, because ``divergent values''\footnote{The thesis' results promote changes in cultural, technological and societal values in order to achieve a transition.} are incorporated without consultation with stakeholders or any other ``democratic'' approach \parencite{rounsevell2010_Developingqualitativescenario}. However, the scientific credibility of the thesis is well supported by the literature used and, in the particular case of the SSP1-MOB narrative, by being backed by the highly acknowledged scenario framework of the IPCC.

With respect to the AUTOLOCK conceptual model, it is clear that the CLDs developed are not the ultimate depiction of the automobility system. They are not even the \emph{only} possible representation of the aspects they are modelling---again, a participatory approach to their development, such as Group Model Building, would incorporate more points of view, eliciting the most important variables and links in the system \parencite{laurenti2014_GroupModelBuilding}. Rather, the AUTOLOCK model should be seen as an example of the potential that CLDs have for (a) describing system feedback structures and (b) conveying that information in an understandable way for decision makers and (transition) researchers alike. Many other perspectives could have been taken to model the automobility system, at this highly abstract level. The approach taken in this thesis for each of the AUTOLOCK model components is meant to provide enough basis for a discussion of the dynamic behaviour of the automobility system. Furthermore, the focus on abstract concepts and the loose system boundary (e.g., the inclusion of ideological discourses in the culture legitimation perspective) try to illustrate the difficulty in dealing with such a complex issue.

Another limit of the thesis lies in the breadth of the system perspective used. It is very complex and burdensome to include all the relevant aspects in a discussion of sustainable futures. For example, the fuel/energy discourse that permeates many transport studies is not explored in-depth in this thesis, nor is the technological dimension, in more general terms (new vehicle technologies, efficiency improvements, etc.). However, the constrained time and resources available to conduct this research do not allow for a further expansion of the system boundaries. Anyhow, it was a deliberate decision to shift the focus of this study onto the socio-cultural dimensions of the mobility system, following the hypothesis formulated in the \nameref{s:intro:aim-objectives} section.

Finally, the lack of (extensive) quantitative figures in either of the methodological perspectives used throughout the thesis (the backcasting process, the CLD model, etc.) poses a potential limitation to the credibility of the results. Scientific (natural sciences), engineering, economics and planning/policy research communities are used to quantitative evaluations and trust the scientific accuracy of studies when formal modelling has been performed. However, the socio-cultural focus of the thesis and the use of narratives, backcasts and transition perspectives actually brings it closer to social sciences. Furthermore, it is the opinion of the author that providing figures in a normative description of mobility futures, without the use of participatory methods, would render the results as non-credible and liable to huge biases.