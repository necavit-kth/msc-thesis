The main line of research that stems from the work in this thesis is the investigation of policy analysis frameworks. This is, researching whether transition theory, and the MLP in particular, can help in the design of policies that fit sustainability better than the current paradigm. In this respect, the scholars from the transitions management field have already made progress, but their approach is a little different. Whereas they focus on niche support and long term goals, this thesis also emphasises the ability of the MLP as an integrative narrative for more ``specific'' assessment tools.

A straightforward extension of the thesis would be to carry out the backcasting process (and the development of the qualitative narrative that serves as the guideline) in a participatory setting. This way, different policy results might be derived or, at the very least, confirmed and validated by stakeholders and general participants. Democracy is, unfortunately, a forgone aspect of sustainability studies. In the case of the thesis, due to time and resources constraints, but in a more general case, due to technocratic views on what science and research is all about. Lengthier study settings, such as PhD research or government officials could undertake this effort.

The usage of Causal Loop Diagrams in combination with other policy design mechanisms is an interesting contribution that deserves further work as well. Most of the literature on System Dynamics and CLDs deals with narrower scopes and bounded systems than what has been studied here. In particular, they are (almost) never used to model ``soft'' issue such as cultural frameworks. Despite the difficulty to quantify these aspects, the argument of this thesis is that CLDs could be used to mentally model these very important issues. They definitely serve the purpose of making mental models explicit, especially when developed in workshops or through Group Model Building techniques.