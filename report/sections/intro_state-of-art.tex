\section{State of the art}
\label{s:intro:state-of-art}

\subsection{Sustainable mobility indicators}
\label{ss:intro:sustainable-mobility-indicators}

\todo[inline]{According to [SOME PAPER], most indicator frameworks are not used as they should: for their instrumental/operational role. Instead, policy makers just use them as another source of knowledge/information, because they claim that ``sets of numbers'' do not convey the necessary insights for policy design or formulation. \textbf{RESEARCH GAP:} Indicators are focused on impact assessment, while policy design should not be focused on a responsive approach -- a proactive, driver-based approach is the key to a sustainable mobility transition. There is a need to model and conceptualize the ``engine'' of the mobility system and, then, find leverage points for effective policy design.}

\subsection{Sustainable socio-technical transitions}
\label{ss:intro:sustainable-transitions}

\todo[inline]{Provide an overview of how do transition studies deal with the issue of unsustainable mobility and, in particular, automobility. What are regimes? What are niches? What are the main approaches? (Section 3.2 in Geels, Kemp et al., 2012)}

\todo[inline]{\textbf{RESEARCH GAP:} To which extent do transition studies address the long term planning of mobility? Transition management does, but might be combined with other planning tools, such as modelling and forecasting and backcasting for long term visions.}