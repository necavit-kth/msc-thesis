%To address all the aforementioned issues, policy packages or simultaneous enforcing of different policies are needed, because of the complexity involved in effectively reducing transportation impacts~\parencite[ch. 3, p. 45]{garciasierra2014_Travelbehaviourenvironmental}. Regulation cannot, however, be designed without evaluating the caused impacts, at any level, in the short and long term -- too many resources and possible negative outcomes would be at stake. \textit{Policy assessment} from the systems thinking perspective is, therefore, a key issue to develop, due to the difficulty of dealing with entire systems, their internal dynamics and the emergent systemic behaviour patterns, such as feedback loops, rebound effects and hidden causalities. In this regard, the field of \textit{system dynamics} can help capture such structures and cause-effect chains \parencite{hjorth2006_Navigatingtowardssustainable}. The holistic nature of system dynamics models can help achieving an integrated assessment framework for urban mobility policies, by delivering information on a set of several indicators, at the environmental, social and economic levels.
%
%Finally, there is another perspective to be considered in the thesis. Policy development in the field of transportation has been mainly focused on two paths, according to \textcite{koehler2009_transitionsmodelsustainable}, to increase the sustainability of the system: (a) efficiency increasing measures, by means of incentives for technological enhancements (e.g. better engines or fuel mixes) and more stringent pollution limits and (b) behavioural change management, i.e., measures aimed at modal shift -- encouraging people to shift from private cars to public transport, for example. Both approaches, albeit successful to some extent, have not delivered the expected results so far, due to the high inertia and stabilization mechanisms inherent to the current dominant regime for transportation: internal combustion engines cars-based mobility \parencite{geels2012_AutomobilityTransitionSocio}. However, a third way is possible, when it comes to policy design for sustainable mobility: \textit{transition management oriented policy} (TMOP).
%
%The approach of TMOP entails adopting a longer term thinking mindset (usually one or several generations), multi-level and multi-domain thinking, maintaining support for a large set of solutions and with a focus in system \textit{innovation} alongside system \textit{improvements} \parencite{rotmans2001_Moreevolutionthan}. Moreover, flexibility in the objectives of policies is encouraged, as well as a more qualitative perspective to policy goals. All these characteristics configure a policy design mindset that could be the key to unlock a true game change in urban mobility, by drifting the focus from quantitative, efficiency measures to a transition and innovation vision, where something more than technological progress is harnessed to achieve a sustainable transport system in the future.

Historically\footnote{The historical time-frame considered in the thesis goes back to the beginning of the 20th century.}, in the ``advanced'' economies of Europe and North-America, personal mobility issues such as congestion and accessibility\footnote{Accessibility is meant as the capacity of reaching (travelling) a destination from ``any'' other given point in a territory.} have been addressed through the development of infrastructure, through increased incentives for automobility and, to some extent, travel demand management \parencite{lyons2012_VisionsFutureNeed}. Policy solutions for environmental social impacts of automobility have also been focused on technological improvement and, on a minor scale, modal shift\footnote{Modal shift refers to changes in the shares of travel modes (reducing car use in favour or biking, for example)} \parencite{koehler2009_transitionsmodelsustainable}.

However effective these policies were in the past---the paradigm of ``predict-and-provide'' for infrastructure development (to address congestion) has already been dismissed in the UK, being regarded as non-efficient and even counter-productive \parencite{goodwin2012_ProvidingRoadCapacity}---, it is clear that they are not so nowadays. Faced with pressing global trends like population growth and the fact that emergent economies in Asia and South-America are also embracing automobility as the paradigm of personal mobility, pressure keeps building on the natural and social environments. A new policy approach is needed; one that is capable of transforming the mobility system into a more sustainable one. Some efforts have been taken to fill this gap, through policy assessment tools like, for example, sustainability indicator frameworks \parencite{castillo2010_ELASTICmethodological,haghshenas2012_Urbansustainabletransportation,litman2007_DevelopingIndicatorsComprehensive,shiau2013_Developingindicatorsystem}.

Despite the best of the intentions behind them, policy assessment tools such as indicators have not performed as expected. Most indicator frameworks are not used as they should, i.e., for their instrumental and operational roles. Instead, policy makers just use them as another source of information, because they claim that ``sets of numbers'' do not convey the necessary insights for policy design or formulation \parencite{gudmundsson2013_SomeuseLittleinfluence}. One of the main drawbacks of indicators is that they are focused on \emph{impact} assessment. Policy design should not be focused on a responsive approach -- a proactive, driver-based approach is the key to a sustainable mobility transition. There is a need to model and conceptualize the ``engine'' of the mobility system and, then, find leverage points for effective policy design. Other traditional policy assessment tools, such as Integrated Assessment Models, are much broader and do evaluate the drivers of the transport sector, but fail to capture the social dimension of the system (they tend to be focused on economic variables and trends such as fuel prices) \parencite{creutzig2015_EvolvingNarrativesLow}. One hypothesis held in this study is that traditional policy assessment tools lack either the systems perspective necessary to avoid policy resistance\footnote{\emph{Policy resistance} refers to unexpected responses of a system to a certain policy measure, usually counter-acting the intended effect.} or a more normative approach to facilitate policy design.

One very important research development in the latest decades has been setting sustainable mobility \emph{visions} for the future. They form the foundation of any further policy or socio-technical development and there are examples of such, like the seminal paper by \textcite{banister2008_sustainablemobilityparadigm}, entitled ``The sustainable mobility paradigm''. However, the \emph{path} from the current situation to the desired vision of mobility remains rather unexplored. Paradigm exploration papers like Banister's provide with general policy recommendations, but do not dive deep into this realm. Additionally, they sometimes fail to account for the dynamic behaviour of the system as a whole and the mechanisms through which policy resistance is created remain rather unexplored. This is, they do not fully investigate the actual reasons why policy is sometimes ineffective, through stability and change dynamics. Moreover, these studies rarely provide with analysis of cultures or of system agents relations, thus being centred in technological, institutional and behavioural\footnote{Note that behaviour is conditioned by culture, but it is not equivalent. Behaviours can be changed within a certain practice space and still remain embedded in the same cultural framework of the previous behaviour.} aspects of mobility.

Finally, a new field of research has emerged that aims to tackle some of the shortcomings discussed in the previous paragraphs: \emph{transition studies} (or \emph{theory}). Scholars like Frank Geels, René Kemp and Jan Rotmans have spearheaded this research community, albeit some differences among their approaches: the tradition of \emph{socio-technical} transitions deals with retrospective and future studies of the changes suffered by socio-technical systems \parencite{geels2001_Technologicaltransitionsas,geels2005_DynamicsTransitionsSocio}, while the tradition of \emph{transition management} is focused on the governance of complex socio-technical systems that are meant to undergo a transition process \parencite{rotmans2001_Moreevolutionthan}. Even though this approach is explained in more detail in the \nameref{c:methods} chapter, it is worth noting that the central focus of the theory is the stability and change dynamics of the systems under study. Following \textcite{geels2001_Technologicaltransitionsas,rotmans2001_Moreevolutionthan}, transition studies investigate the co-evolution processes and multi-dimensional interactions occurring among agents (users, companies, policy makers, markets, culture, etc.) in a system. It is, therefore, a dynamics-centric, system-wide and possibly normative perspective that aims to understand how and why transitions take place in socio-technical systems.
