%To address all the aforementioned issues, policy packages or simultaneous enforcing of different policies are needed, because of the complexity involved in effectively reducing transportation impacts~\parencite[ch. 3, p. 45]{garciasierra2014_Travelbehaviourenvironmental}. Regulation cannot, however, be designed without evaluating the caused impacts, at any level, in the short and long term -- too many resources and possible negative outcomes would be at stake. \textit{Policy assessment} from the systems thinking perspective is, therefore, a key issue to develop, due to the difficulty of dealing with entire systems, their internal dynamics and the emergent systemic behaviour patterns, such as feedback loops, rebound effects and hidden causalities. In this regard, the field of \textit{system dynamics} can help capture such structures and cause-effect chains \parencite{hjorth2006_Navigatingtowardssustainable}. The holistic nature of system dynamics models can help achieving an integrated assessment framework for urban mobility policies, by delivering information on a set of several indicators, at the environmental, social and economic levels.
%
%Finally, there is another perspective to be considered in the thesis. Policy development in the field of transportation has been mainly focused on two paths, according to \textcite{koehler2009_transitionsmodelsustainable}, to increase the sustainability of the system: (a) efficiency increasing measures, by means of incentives for technological enhancements (e.g. better engines or fuel mixes) and more stringent pollution limits and (b) behavioural change management, i.e., measures aimed at modal shift -- encouraging people to shift from private cars to public transport, for example. Both approaches, albeit successful to some extent, have not delivered the expected results so far, due to the high inertia and stabilization mechanisms inherent to the current dominant regime for transportation: internal combustion engines cars-based mobility \parencite{geels2012_AutomobilityTransitionSocio}. However, a third way is possible, when it comes to policy design for sustainable mobility: \textit{transition management oriented policy} (TMOP).
%
%The approach of TMOP entails adopting a longer term thinking mindset (usually one or several generations), multi-level and multi-domain thinking, maintaining support for a large set of solutions and with a focus in system \textit{innovation} alongside system \textit{improvements} \parencite{rotmans2001_Moreevolutionthan}. Moreover, flexibility in the objectives of policies is encouraged, as well as a more qualitative perspective to policy goals. All these characteristics configure a policy design mindset that could be the key to unlock a true game change in urban mobility, by drifting the focus from quantitative, efficiency measures to a transition and innovation vision, where something more than technological progress is harnessed to achieve a sustainable transport system in the future.

Historically\footnote{The historical time-frame considered in the thesis goes back to the beginning of the 20th century.}, in the ``advanced'' economies of Europe and North-America, personal mobility issues such as congestion and accessibility\footnote{Accessibility is meant as the capacity of reaching (travelling) a destination from ``any'' other given point in a territory.} have been addressed through the development of infrastructure, through increased incentives for automobility and, to some extend, travel demand management \parencite{lyons2012_VisionsFutureNeed}. Policy solutions for environmental social impacts of automobility have also been focused on technological improvement and, to some extent, modal shift\footnote{Modal shift refers to changes in the shares of travel modes (reducing car use in favour or biking, for example)} \parencite{koehler2009_transitionsmodelsustainable}.

However effective these policies were in the past --- the paradigm of ``predict-and-provide'' for infrastructure development (to address congestion) has already been dismissed in the UK, being regarded as non-efficient and even counter-productive \parencite{goodwin2012_ProvidingRoadCapacity} ---, it is clear that they are not so nowadays. Faced with pressing global trends like population growth and the fact that emergent economies in Asia and South-America are also embracing automobility as the paradigm of personal mobility, pressure keeps building on the natural and social environments. A new policy approach is needed; one that is capable of transforming the mobility system into a more sustainable one. Some efforts have been taken to fill this gap, through policy assessment tools like, for example, sustainability indicator frameworks \parencite{castillo2010_ELASTICmethodological,haghshenas2012_Urbansustainabletransportation,litman2007_DevelopingIndicatorsComprehensive,shiau2013_Developingindicatorsystem}.

Despite the best of the intentions behind them, policy assessment tools such as indicators have not performed as expected. Most indicator frameworks are not used as they should, i.e., for their instrumental and operational roles. Instead, policy makers just use them as another source of information, because they claim that ``sets of numbers'' do not convey the necessary insights for policy design or formulation \parencite{gudmundsson2013_SomeuseLittleinfluence}. One of the main drawbacks of indicators is that they are focused on \emph{impact} assessment. Policy design should not be focused on a responsive approach -- a proactive, driver-based approach is the key to a sustainable mobility transition. There is a need to model and conceptualize the ``engine'' of the mobility system and, then, find leverage points for effective policy design. Other traditional policy assessment tools, such as Integrated Assessment Models, are much broader and do evaluate the drivers of the transport sector, but fail to capture the social dimension of the system (they tend to be focused on economic variables and trends such as fuel prices) \parencite{creutzig2015_EvolvingNarrativesLow}.

and through developing visions of what a sustainable mobility system would look like.

However, the path from the current situation to the desired vision of mobility remains rather unexplored. Transition theories are meant to fill this later gap.

\todoparagraph{Describe the efforts so far to define the concept/paradigm of sustainable mobility. Especially, focus on the insights derived from \textcite{banister2008_sustainablemobilityparadigm} and similar papers (is there anything in the Geels 2012 book?)}

\todoparagraph{Concerning the ideas in \textcite{banister2008_sustainablemobilityparadigm}, talk about the 4 key elements to develop (in policy-ish) and the highlight of stakeholder participation as the foundation. Regarding this, there are some pairs of opposing forces (personal utility vs social welfare), (active involvement vs passive persuasion) that deserve to be mentioned.}

\todoparagraph{\textcite{banister2008_sustainablemobilityparadigm} does stress that the behavioural aspects of mobility deserve much further attention, and presents the notion of public acceptability as a key to effective policy introduction, but still misses (not too much, admittedly) the point of \textit{culture} and intrinsic moral/personal values in the automobility conceptualisation of our society. This deserves as well a lot of attention.}

\subsection{Socio-technical transitions}
\label{ss:intro:transitions-theory}

\todoparagraph{Provide an overview of how do transition studies deal with the issue of unsustainable mobility and, in particular, automobility. What are regimes? What are niches? What are the main approaches? (Section 3.2 in Geels, Kemp et al., 2012) Highlight that these studies actually do stress the cultural component of automobility. Without acknowledging it, there is no way that a transition is successful and, even less, that it can be managed/accelerated/supported by policy means.}

\todoparagraph{\textbf{RESEARCH GAP:} To which extent do transition studies address the long term planning of mobility? Transition management does, but might be combined with other planning tools, such as modelling and forecasting and backcasting for long term visions.}