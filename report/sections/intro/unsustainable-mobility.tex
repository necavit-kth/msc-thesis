Several issues of the current mobility system make it qualify as unsustainable. Direct downstream impacts, e.g. greenhouse gas emissions, are the most directly perceivable hazards, but it is their combination with global future trends that increases the significance of mobility impacts on sustainability. Issues such as population growth \parencite{un-desa2015_WorldPopulationProspects,kc2017_humancoreshared}, peak oil \parencite{kerr2011_Peakoilproduction}, expected impacts from climate change and growing economies in Asia, South-America and Africa, all highlight the acceleration and exponential expansion of the negative effects that high mobility poses to the environment, the economy and to human health and social systems.

Transport related airborne pollution is one of the main causes of respiratory diseases and associated increase in morbidity in densely populated areas~\parencite{vimercati2011_Trafficrelatedair,who2006_Airqualityguidelines}. Ambient air pollution is estimated to cause 4.4 million premature deaths around the globe~\parencite{forouzanfar2016_Globalregionalnational} and the link from air pollution to both severe health problems and high traffic volumes is well known and thoroughly researched~\parencite{who2006_Airqualityguidelines}: \ce{NO_x} emissions that lead to increases in \ce{PM_{2.5}} particulate and ozone concentrations are directly linked to diesel combustion engines, in heavy duty but also light duty vehicles \parencite{anenberg2017_Impactsmitigationexcess}. The fact that regulations and emission limits are in place within the automotive industry has not alleviated the problem, due to ever-growing automobile use and because of the industry efforts to deceive such regulations, avoiding costly research and development investments, as is the case of the recent ``dieselgate'' scandal \parencite{guardian2017_Volkswagenrevealsrecord}.

The current unsustainable mobility system not only causes respiratory health issues, but also congestion, accidents, noise pollution, infrastructure degradation and, finally, it is one of the sectors that most contribute to climate change \parencite{korzhenevych2014_UpdateHandbookExternal}. Congestion is, for example, the cause of massive costs in terms of reduced productivity, increased energy and fuel consumption, higher accident risk and its subsequent economic impacts which, for instance, are estimated at 4.2\% of Beijing's 2010 GDP\footnote{Gross Domestic Product} ~\parencite{li-zeng2012_SocialCostTraffic}. Road-related accidents alone (cars, buses and other vehicles aggregated) cause vehicle losses and damages and, most importantly, death rates that exceed 1.2 million worldwide per year or \num{28077} in the European Union (EU) in 2015 \parencite{who2017_GlobalHealthObservatory}. Safety in rail and aviation is much higher (especially per person kilometre travelled) than in roads, but they still take away 993 lives in rail-related accidents and 155 in aviation (EU data, 2015) \parencite{eurostat2017_StatisticsExplainedRailway,eurostat2017_EurostatOnlineDatabase}.

The transport sector is responsible for 27.8\% of the global final energy consumption (2014 data), with over 95\% of this energy coming from fossil sources (oil, primarily, but natural gas and coal too) \parencite{iea2017_Statisticswebportal}. \textcite{chapman2007_Transportclimatechange} already estimated that 26\% of the total world's \ce{CO_2} emissions were borne in the transport sector, of which 65\% are originated in road transport. Given the enormous pressure that climate change puts on the resilience of modern societies \parencite{ipcc2014_ClimateChange2014} and the current undertaking to tackle this global challenge --- take, for example, the recent Paris Agreement Framework Convention on Climate Change \parencite{clemencon2016_TwoSidesParis} ---, transport (mobility) is one of the sectors that must be thoroughly examined, revised and challenged to deliver urgent greenhouse gas emissions mitigation.

\subsection{Automobility at the core}
\label{ss:intro:automobility-at-core}
Automobility, as a personal mobility solution, has brought about many positive consequences, from the individuals point of view. However, its exponential growth and reliance on fossil fuels and massive infrastructures to work deem it as a global threat to the environment and, at a more local scale, to the quality of life of the very same individuals that make use of this transport mode. Given that automobility accounts for almost 36\% of the travel demand (in person kilometres per year) \parencite{vuuren2017_Energylanduse}, and that 65\% of \ce{CO_2} emissions from all travel modes are attributed to road transport \parencite{chapman2007_Transportclimatechange}, it is logical to state that automobility is, indeed, a major player in the sustainable mobility discussion. Therefore, this thesis will place a good deal of emphasis on this particular mode. The study scope encompasses the relation between this regime and public transport, or between automobility and urban planning, for example --- limiting the scope of this research to the automobile system would certainly not be sufficient to address the broad concept of sustainable mobility.

%\subsection{The need for a transition}
%\label{ss:intro:need-for-transition}
%
%\todoparagraph{Introduce the necessity of a transition (because impacts and expected growth in those due to population/affluence growth, etc. etc.)}
%
%There is therefore, at the environmental, social and economic levels, a societal need for a transition away from the current dominant regime of automobility. A new sustainable mobility paradigm is to emerge if we are truly committed to a sustainable future. How this paradigm may look like is not clear and needs further investigation from a future studies perspective.