The previous sections have covered the impacts caused by the current mobility paradigm, as well as some of the policy and research approaches to solving the damaging effects of the transport system. The major gap identified in the \nameref{s:intro:state-of-art} section is the poor performance of policy (and research) to deliver the expected improvements in the mobility sector. It is somewhat clear that there is a huge resistance to policy efforts and that research has been, so far, unable to identify its ultimate causes --- at the very least, if the identification has ever been positive, policy makers have been unable to introduce the appropriate measures to tackle the problem. A \emph{primary research question} remains thus open:
\blockquote{\emph{How can modern societies develop a sustainable mobility system?}}

This broad question actually entails two specific issues to be addressed jointly: (a) \emph{what} is a ``sustainable'' mobility system exactly (how is it defined) and (b) \emph{how} to reach (develop) such a sustainable system, from the current configuration. While there is considerable amounts of literature on the topic of paradigm definition of sustainable mobility \parencite[see, e.g.,]{banister2000_Europeantransportpolicy,banister2008_sustainablemobilityparadigm,hoeyer2000_SustainableTourismSustainable,burns2013_Sustainablemobilityvision}, there is not that much regarding the actual development pathway to it, at least from a normative and/or social point of view. As reviewed in \sref{s:intro:state-of-art}, a lot of the literature on pathways to sustainable mobility deals with efficiency improvements of specific technologies (especially that of automobiles) and on infrastructure optimisation \parencite{creutzig2015_EvolvingNarrativesLow,koehler2009_transitionsmodelsustainable,lyons2012_VisionsFutureNeed}

The conjecture is that, regarding the historically observed policy resistance demonstrated by unsustainable mobility, none of these approaches challenges the fundamentals of the status quo of mobility. In particular, it is cultural frameworks and sociological interactions at multiple levels that have remained outside of the scope of policy making and policy advise in the past. The main \emph{hypothesis} of this study can thus be formulated as:
\blockquote{\emph{Policy and research efforts have, so far, failed to account for and challenge the socio-cultural framework that underpins unsustainable mobility.}}

The primary research question will therefore be answered from the point of view of the stated hypothesis. By (1) drawing on the integrated socio-technical narrative\footnote{Narrative is meant here as the discursive framework that a theory or science provides, as a way to structure explanations from a certain perspective.} that transition studies enable and (2) acknowledging the importance of automobility as a key driver of impacts in the mobility system, the \emph{aim of this thesis} is:
\blockquote{\emph{To investigate how policy can help to achieve a sustainable mobility system in the future, by studying the \emph{dynamics} of a \emph{socio-technical} \emph{transition} away from the current dominant regime of automobility.}}

The focus of this study is not so much on the concept of sustainable mobility, but on the \emph{transition} pathway to it. It is not in technological improvements or demand management techniques either, but on the socio-technical and cultural \emph{drivers} of (auto)mobility to understand what are the elements of resistance and enablers of change in the mobility system. Regarding the \emph{target audience} of the thesis, the intention is to develop insights for governmental policy makers\footnote{At either local, regional, national or international levels.} and other stakeholders capable of decision making in the field of mobility, from the transition studies perspective --- the ultimate goal of the thesis is the derivation of policy recommendations, informed by a transitions theory perspective.

Finally, this investigation of sustainable futures of mobility requires that the following sub-goals (objectives) are completed:
\begin{enumerate}[leftmargin=*,label=\textbf{Obj.~\arabic*.}]
\item Develop a description of a desirable future mobility system, this is, a \emph{vision} that acts as the final goal of a transition process. The vision must incorporate socio-cultural, as well as technological and developmental, aspects of mobility.
\item Analyse the \emph{changes} that separate the desired distant future from the current mobility system. These changes must reflect the actual requirements for the transition to happen.
\item Analyse the current mobility system to identify (a) its socio-cultural \emph{mechanisms of resilience} (policy resistance) and (b) the \emph{opportunity windows} for introducing changes to achieve a successful transition away from it.
\item Develop long-term policy recommendations to implement the required changes for a transition that have been previously identified. Policy recommendations must be coherent with the transition goal itself, but also among them.
\end{enumerate}