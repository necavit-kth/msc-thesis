The previous (sub)sections have covered the harmful and beneficial impacts caused by the current mobility paradigm, as well as some of the policy and scientific approaches to the solutions designed to mitigate the damaging effects of the transport system. For the sake of emphasis, automobility is, as already mentioned, the main focal point of research in terms of impact alleviation, due to its dominant position in terms of travel volume. Several gaps have also been identified, not only in terms of scientific and policy research, but also between the expected outcome of the implemented policies and their actual results (impossible demand meeting by infrastructure development, insufficient efficiency improvements, inefficient demand management schemes, etc.). This lack of major success in the scientific and political endeavours to reach a more sustainable mobility leaves a \textbf{primary research question} still open:
\blockquote{``\textit{What is} a sustainable mobility system and \textit{how} can modern societies develop the necessary adaptations to reach it?''}

%TODO rephrase the whole following paragraph
This broad question actually entails two separate but intimately related issues: (a) the definition of the \textit{concept} of sustainable mobility itself and (b) the \textit{path} from the current situation to the targeted (future) sustainable transport system. \textbf{SOMETHING IS MISSING HERE, BECAUSE IT NO LONGER BELONGS HERE!}. The second issue has been tackled in different ways because of the inherent dependency on what is understood for sustainable mobility --- i.e., the first issue --- and because different disciplines adopt different epistemic approaches \parencite{creutzig2015_EvolvingNarrativesLow}. As has been previously reviewed, a lot of attention had traditionally been brought upon technological enhancements of travel modes (especially for cars) and on infrastructure optimisations. All of these did not, however, challenge the underlying dominance of the automobile regime in the current mobility system and did not, therefore, aim for a transition to a different regime or a radical change in the way we understand personal mobility.

More recently, the concept of a sustainable mobility paradigm has evolved with an increasing understanding that automobility is one of the key causes of the negative impacts of the system. Authors and institutions have thus embraced a more proactive approach to reduce the need for automobility, especially through demand management. However, the system's inertia and resistance to change has become evident, resulting in failures to implement more drastic demand reducing policies \parencite{geels2012_AutomobilityTransitionSocio}. The hypothesis held by the scholars in the field of \textit{transition studies} (see \sref{s:intro:state-of-art}), and which this thesis shares, is that the majority of efforts made so far to reduce the ever growing impacts of mobility have failed to take into account the \textit{socio-cultural} dimension of the very concept of mobility.

Therefore, by (1) building upon the principles of sustainable mobility presented by \textcite{banister2008_sustainablemobilityparadigm}, (2) drawing on the integrated socio-technical narrative that transition studies enable and (3) acknowledging the importance of automobility as the key driver of impacts in the urban transport system, the \textbf{aim of this thesis} is:
\blockquote{To investigate how to achieve a sustainable mobility system in the future, by studying the dynamics of a socio-technical \textit{transition} away from the current dominant regime of automobility.}
This is, the focus of this study will not be on the concept of sustainable mobility, but on the \textit{transition} pathway to it and not in technological improvements or demand management techniques either, but on the socio-technical and cultural \textit{drivers} of (auto)mobility to understand what are the elements of resistance to and enablers of change in the transport system.

Regarding the target audience of the thesis, the purpose of the study is to develop insights for policy makers and other stakeholders capable of decision making in the field of mobility from the transition studies perspective. This discipline has already tackled the problem (see, for example, \textcite{geels2012_AutomobilityTransitionSocio}) but lacks, in the opinion of the author, the informative power that other policy assessment tools have. This means that, however well transition studies capture the socio-cultural aspects of the (lack of) transition in the mobility system, the nature of its epistemological background --- social sciences --- makes it more complex for key stakeholders to grasp their results. Therefore, \textbf{another important goal} of the thesis is to incorporate the transition studies perspective and discourse into more traditional and accessible policy assessment tools.

\todoparagraph{Is the study normative? In which way is it? Are there data assumptions? Is there assumptions on the future development of the world? (scenarios, etc.)}
\todoparagraph{What is the normative assumption? Possibilities: \textit{liberal} or \textit{welfarist}, according to Creutzig (2015). Alternative: \textit{modular assessment} model.}

To sum up, the following set of objectives correspond to the tasks that make up this investigation of sustainable futures of mobility:
\begin{enumerate}[leftmargin=*,label=\textbf{Obj.~\arabic*.}]
\item\label{obj:1} Analyse the transition related dynamics of the socio-cultural and technical drivers for the current automobility dominated system, in order to assess both the system's resistance to and the possibilities for change.
\item Discuss and develop a description of a desirable future mobility system. This description should be comprehensive enough to incorporate both the drivers for mobility and their downstream effects.
\item Analyse the necessary changes that separate the distant desired future from the current situation, from the point of view of transition studies.
\item Develop long term policy recommendations that are compliant with the transition based changes that have been previously identified. The appraisal of the transition dynamics in Objective 1 is incorporated to improve the insights for policy design.
\end{enumerate}

\subsection{Automobility at the core}
\label{ss:intro:automobility-at-core}
Automobility, as a personal mobility solution, has brought about many positive consequences, from the individuals point of view. However, its exponential growth and reliance on fossil fuels and massive infrastructures to work deem it as a global threat to the environment and, at a more local scale, to the quality of life of the very same individuals that make use of this transport mode. Given that automobility accounts for almost 36\% of the travel demand (in person kilometres per year) \parencite{vuuren2017_Energylanduse}, and that 65\% of \ce{CO_2} emissions from all travel modes are attributed to road transport \parencite{chapman2007_Transportclimatechange}, it is logical to state that automobility is, indeed, a major player in the sustainable mobility discussion. Therefore, this thesis will place a good deal of emphasis on this particular mode. The study scope encompasses the relation between this regime and public transport, or between automobility and urban planning, for example. Limiting the scope of this research to the automobile system would certainly not be sufficient to address the broad concept of sustainable mobility.