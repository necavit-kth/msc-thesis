\section{Unsustainable mobility}
\label{s:intro:unsustainable-mobility}

\todo[inline]{This section should describe the \textbf{reference mode} (Sterman, 2000): the set of graphs and descriptive data showing the development of the problem over time. \textit{Consider moving the \textit{reference mode} discussion to the Results chapter, wherever the Sys. Dyn. model is introduced.}}

\todo[inline]{Explain the situation of the transport system, with regards to sustainability issues: environmental hazards, (un)economic performance and social impacts of the system as it is. Also describe the socio-economic benefits (increased economic output due to high mobility, etc.) but highlight that mobility is envisioned as a ``right'' when it is not so, necessarily.}

\todo[inline]{Frame the mobility system and its impacts and developments within the global pressing trends of increasing population, affluence and so on.}

\subsection{Mobility regimes}
\label{ss:intro:mobility-regimes}

\todo[inline]{Introduce the concept of socio-technical/technological regimes. Describe, in more detail, automobility as such a regime and compare it to other mainstream transport solutions (regimes): aviation, rail, bus and cycling/walking.}