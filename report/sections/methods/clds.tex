The third objective of the thesis is the analysis of policy resistance mechanisms of the current mobility system. The chosen method to do this is the first of the steps used in the broader methodological framework of System Dynamics: Causal Loop Diagrams (CLDs). Due to the difficulty of dealing with entire systems, their internal dynamics and the emergent systemic behaviour patterns, such as feedback loops, rebound effects and hidden causalities, simple linear/mechanistic (conceptual) models are simply insufficient to provide a complete system picture. In this regard, the field of System Dynamics can help capture such structures and cause-effect chains (Hjorth and Bagheri, 2006). An interesting and relevant example of System Dynamics modelling is the World3 model in ``The Limits to Growth'' report, from the Club of Rome, where global trends were assessed to derive prospects of the future regarding population and economic growth, fossil fuel consumption, etc. \parencite{meadows1972_LimitsGrowthReport}.

CLDs consist of variables linked by causal relations, which can either be positive (directly proportional) or negative (inversely proportional). Note that these causal links are not quantified in a CLD: the quantification step (attaching equations to the relations) is performed in a further modelling stage, giving rise to stock-flow models \parencite{sterman2000_BusinessDynamics}. An important concept is necessary to interpret the diagrams in the \nameref{c:results} chapter: the feedback loops. These structures are cycles of causes and effects that are the actual reason behind the dynamic behaviour of systems. These feedback mechanisms can either reinforce or balance the behaviour of the system (at least that of the variables included in a loop). Reinforcing loops are responsible for the exponential growth (or shrinkage) of the involved variables, whereas balancing loops orient the variables asymptotically towards a ``target'' level \parencite{sterman2000_BusinessDynamics}.

With regards to how the CLDs were developed in this thesis specifically, the methodology followed was very similar to the process described by \textcite{laurenti2015_TowardsAddressingUnintended}. A first step was taken to ``frame the challenge'', this is, to decide what was the problem or issue to tackle. In the case of this thesis, the goal was to identify feedback structures within the urban planning and cultural dimensions of mobility that reinforce and stabilise automobility. After this, a core conceptual model was developed, expanding the system boundaries to capture enough causal structures to understand the feedback structure. The final step was to ``prune'' the model, shrinking the boundaries so that it became simpler and easier to convey.
