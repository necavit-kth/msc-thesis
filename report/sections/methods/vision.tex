Concerning the first of the objectives described in the Introduction chapter, no formal method was used to \emph{obtain} the ``vision'' of a future mobility system. Indeed, a literature review was performed, searching for papers with the keywords ``sustainable'', ``mobility'' and ``paradigm''. While several articles were found (e.g., \textcite{banister2008_sustainablemobilityparadigm}), little or none of them actually gave a clear picture of a \emph{future paradigm} of sustainable mobility. Therefore, the decision was taken to build the vision on an already existing long-term scenario framework by the \gls{IPCC} (see \sref{s:results:ssp1-mob} for more detail). The vision was developed following the same principles as \textcite{oneill2017_roadsaheadNarratives} did for the IPCC scenarios: a \emph{qualitative narrative} in which the aspects of the sustainable mobility paradigm are explained qualitatively.

The fact that the IPCC scenarios are developed from a set of \emph{narratives} actually makes them easier to transform into a ``vision'' than more traditional scenarios: the IPCC scenarios explore final \emph{states}, rather providing with forecasts. Drawing on the scenario typology built by \textcite{boerjeson2006_Scenariotypestechniques}, IPCC's would fall under the ``exploratory'' scenarios category, while the actual approach of the backcasting process (see \sref{s:methods:backcasting}) in this thesis would fall under the ``normative'' category. Despite these seem not to fit, the truth is that in order to build the vision for the backcasting in \ssref{ss:results:ssp1-mob-development}, it is very useful to start from the ground of an already developed, consistent and acknowledged exploration of a \emph{possible} future. This way, a lot of assumptions are already justified and the whole vision frame is not purely speculative in nature.