In order to obtain the desired pathway of the transition to the vision (Obj. 2), the \emph{backcasting} method was used. Backcasting processes are the inverse of traditional forecasts. Instead of extrapolating current trends into the future, they start from a defined end-state or goal that represents a desirable and somehow plausible future. Then, either through stakeholder participatory workshops, expert panels or ``in-house'' development processes, a set of intermediate goals are set at whatever time-frames are required. The backcasting process is finalised when the identification of the trends and changes that separate the desired future and the current state of the system under study \parencite{dreborg1996_Essencebackcasting}. Backcasting usually serves a normative purpose and, most importantly, they are usually rhetorical instead of analytical. The strength of this methodology lies in the ability to extend the range of possibilities under consideration \parencite{mcdowall2006_Forecastsscenariosvisions}.

The main reason behind this choice was the fact that backcasting is, in itself, a normative methodology for futures analysis \parencite{boerjeson2006_Scenariotypestechniques,mcdowall2006_Forecastsscenariosvisions}, something required for filling the gaps identified by this thesis (and to help in the derivation of the policy recommendations for Obj. 4). Additionally, backcasting is also a more promising method when great changes are expected or required, which is the case of a transition to sustainable mobility \parencite{hoejer2000_Determinismbackcastingfuture}. Note that, due to resource limitations and time budgets, the backcasting process was not performed in a participatory manner. No stakeholders were involved, nor an expert panel, thus becoming a limitation in the power of the study, as discussed in \cref{c:discussion}.