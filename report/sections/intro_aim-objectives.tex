\section{Aim and objectives}
\label{s:intro:aim-objectives}

As stated in \autoref{s:intro:need-for-transition}, there is, at the environmental, social and economic levels, a societal need for a transition away from the current dominant regime of automobility. A new sustainable mobility paradigm is to emerge if we are truly committed to a sustainable future. How this paradigm may look like is not clear and needs further investigation from a future studies perspective.

The primary research question is: \textit{What are the alternatives for a transition away from the dominant automobility regime?} In order to answer this broad and open question, new methodological frameworks to investigate the possible future development paths of personal mobility are needed. This is so because information tools for policy makers such as indicator frameworks have failed to enable long-term and comprehensive system innovation in the transport sector, for example. A broader perspective is required to study the issue, since the transition entails not only technological or environmental aspects, but cultural and socio-political ones too. Transition studies provide, on one hand, with the appropriate framework to include said cultural and social components, but fail, on the other hand, to inform policy makers with the type of information that they need in order to judge the decisions to be made. This is, transition studies capture socio-cultural aspects of mobility well, but they are not so well suited to convey the information and knowledge they produce to the key stakeholders that play a part in the transition to a more sustainable mobility.

The aim is to ``expand'' more traditional policy assessment tools with the perspective of transition studies to enable a better understanding of the problems of the current mobility system and its potential solutions. Therefore, the idea is to frame these more conventional tools, which convey information in an easier way, within the conceptual framework of transition studies. This way, the problems can be reformulated and ``attacked'' from a new perspective -- one in which the deep cultural roots of the current dominant regime can be analysed and challenged to find alternative paths of development in the future.

The combination of assessment and planning tools, embedded in the discourse of transition studies, is used to discuss possible global patterns of development, by means of a dual backcasting and forecasting approach: scenario backcasting on one hand and system dynamics modelling on the other.

\todo{Move this to results/discussion}Change is to be brought upon by two forces: a different conceptualisation of (personal) mobility and the changes in infrastructure, transport provision and life style that would accompany such change.