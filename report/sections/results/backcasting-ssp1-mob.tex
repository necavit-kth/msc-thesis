The SSP1-MOB vision of the future described in \sref{s:results:ssp1-mob} is used in the following sections to synthesise the changes that the mobility paradigm must undergo\footnote{The changes are not ``suffered'' by an autonomous and disconnected system of mobility, but are introduced by all the agents that play a part in it: users, industries, governments, decision makers, researchers, planners, etc.} to reach the desired form. The synthesis is done using backcasting approach, for which an ``intermediate step'' narrative is presented in \ssref{ss:results:backcasting-2050-intermediate-step}, in a similar manner to SSP1-MOB, but for the year 2050. The changes between the 2100 and 2050 storylines are given in \ssref{ss:results:backcasting-2100-2050}, while the ones between the 2050 narrative and the current situation (2017) are provided in \ssref{ss:results:backcasting-2050-2017}. Finally, a discussion and a summary of the entire backcasted path is presented in \ssref{ss:results:backcasting-the-path}.

\subsection{SSP1-MOB 2050: an intermediate step to sustainable mobility}
\label{ss:results:backcasting-2050-intermediate-step}
\todoparagraph{Provide with a shorter narrative for 2050. Add the summary table of the narrative, as in \nameref{ss:results:ssp1-mob-development} (\sref{ss:results:ssp1-mob-development})}

\blockquote{\sffamily \textbf{SSP1-MOB narrative (2050)}\\ \lipsum[1-5]
}

\begin{table}
\centering
\caption[SSP1-MOB (2050) qualitative variables]{Qualitative variables and trends underlying to the SSP1-MOB narrative for the year 2050.}
\label{t:ssp1-mob-2050-narrative-thesis}
\scriptsize
\begin{tabular}{p{2.5cm}p{3cm}p{8cm}}
\toprule
Category & Variable or feature & Trend or status \\
\midrule
\multicolumn{3}{c}{\textbf{TODO: fill in this table (same information as in 2100's narrative)}} \\
\bottomrule
\end{tabular}
\end{table}

\subsection{Backcasting: 2100 to 2050}
\label{ss:results:backcasting-2100-2050}
\todoparagraph{Quick description (table? itemize?) of the changes occurring between the 2050 and 2100 narratives.}

\subsection{Backcasting: 2050 to 2017}
\label{ss:results:backcasting-2050-2017}
\todoparagraph{Quick description (table? itemize?) of the changes occurring between the 2050 narrative and 2017 baseline.}

\subsection{The backcasted path to SSP1-MOB}
\label{ss:results:backcasting-the-path}
\todoparagraph{Explain and contextualise all the backcasted changes: what is the main path to the sustainable future mobility paradigm?}
As already highlighted in the \sref{ss:results:ssp1-mob-paradigm}, one of the key preconditions for the transition to a sustainable mobility paradigm is a change in the \emph{cultural perception} of mobility (how is it understood). This shift of cultural perspective happens not only at the user level, but also at the urban/traffic planner one. \textcite{banister2008_sustainablemobilityparadigm}, drawing from \textcite{marshall2001_challengesustainabletransport}, presents some of the foundation stones of the change to the sustainable mobility paradigm, with which the SSP1-MOB narrative is aligned, and also provides some insight into the cultural contrasts between such a paradigm and the current mobility culture:
%
\begin{enumeratealpha}
\item Instead of speeding up traffic, slowing down traffic is the new target for both users and planners.
\item Streets are seen as a space for urban life, rather than just public spaces occupied by private automobiles only.
\item Social and environmental multicriteria analyses regarding mobility are performed in addition the already dominant economic assessments.
\item Larger travel times become acceptable, in contrast to the current ever accelerating pace of society and mobility.
\item Attention is shifted from vehicles to people: human, personal mobility is the core of the frame, opposed to just vehicle mobility.
\end{enumeratealpha}

\todonote{finish this}