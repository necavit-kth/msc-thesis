The SSP1-MOB vision of the future described in \sref{s:results:ssp1-mob} is used in the following sections to synthesise the changes that the mobility paradigm must undergo\footnote{The changes are not ``suffered'' by an autonomous and disconnected system of mobility, but are introduced by all the agents that play a part in it: users, industries, governments, decision makers, researchers, planners, etc.} to reach the desired form. The synthesis is done using backcasting approach, for which an ``intermediate step'' narrative is presented in \ssref{ss:results:backcasting-2050-intermediate-step}, in a similar manner to SSP1-MOB, but for the year 2050. The changes between the 2100 and 2050 storylines, along with the ones between the 2050 narrative and the current situation (2017) are then provided in \ssref{ss:results:backcasting-the-path}

\subsection{SSP1-MOB 2050: an intermediate step to sustainable mobility}
\label{ss:results:backcasting-2050-intermediate-step}
\todoparagraph{Provide with a shorter narrative for 2050. Add the summary table of the narrative, as in \nameref{ss:results:ssp1-mob-development} (\sref{ss:results:ssp1-mob-development})}
A particular \emph{vision} of a future sustainable mobility system is outlined in SSP1-MOB. Note that it is not comprehensive description of the myriad of elements composing the system, because it would not be feasible and too many sources of uncertainty would be introduced. Rather, it deals with travel patterns and which travel modes are most common, provided the necessary elements and configurations that make these patterns possible. To understand the frame of this narrative, it is \textbf{highly suggested} to read the original SSP1 storyline, which SSP1-MOB extends, reproduced in \aref{a:ssp1-original-narrative}. The following is the narrated version of SSP1-MOB, describing the global situation of the mobility system in the year 2100:
%
\blockquote{\sffamily \textbf{SSP1-MOB narrative (2050)}\\Due to the 
}

\begin{landscape}
{\scriptsize
\begin{longtable}{p{2.5cm}p{3.5cm}p{6cm}p{6cm}}
\caption[Comparison of SSP1-MOB qualitative variables (2050 vs 2100)]{Comparison of qualitative variables underlying to the 2050 and 2100 SSP1-MOB narratives.}\\
\toprule
& & \multicolumn{2}{l}{Trend or status}\\
\cmidrule(l){3-4} Category & Variable or feature & SSP1-MOB 2050 & SSP1-MOB 2100\\
\midrule
\endfirsthead
\caption*{(\emph{continued}) Comparison of SSP1-MOB qualitative variables (2050 vs 2100)}\\
\toprule
& & \multicolumn{2}{l}{Trend or status}\\
\cmidrule(l){3-4} Category & Variable or feature & SSP1-MOB 2050 & SSP1-MOB 2100\\
\midrule
\endhead
\bottomrule
\endfoot
\bottomrule
\endlastfoot
\label{t:ssp1-mob-2050-narrative-thesis}
\textit{Development scenario} & Societal sustainability awareness & Medium & High \\*
 & Travel demand per capita (pkm/yr) & Similar to the baseline (2017) & Lower than the baseline (2017) \\*
 & Total travel demand (pkm/yr) & Higher than the baseline (2017) & Higher than the baseline (2017) \\*
 & Consumption patterns & Lower consumption levels in HICs, increased consumption in LICs & Less carbon (travel) intensive consumption; lower consumption levels \\*
 & Carbon intensity & Medium; power grid is increasingly decarbonized, but not completely & Low; the system is as decarbonized as possible (electrification) \\*
 & IT access & Widespread in HICs; increasing use of IT services for teleworking, car sharing, intermodal travel, \ldots & Widespread, high speed and capacity networks; IT services for teleworking, car sharing, intermodal travel, \ldots \\\addlinespace
\textit{Land use (urban development)} & Urban density & Medium-high and increasing & High (higher than 2017 baseline) \\*
 & Land use patterns & Mixed-use development paradigm & Mixed-use development paradigm \\*
 & Economic centralisation & High; big cities accumulate a big share of the activity & Medium; cities are hotspots, but jobs are spread amongst them \\*
 & City sizes & Medium to large; megacities and (sub)urban sprawl beginning to shrink & Medium; avoidance of megacities or (sub)urban sprawl \\ \addlinespace
\textit{Travel modes share} & Intermodal travel & Facilitated, but still not common & Facilitated, high acceptancy and usage \\*
 & Public transport (rail, bus, aviation) & Increasing demand supply; higher than the baseline (2017) & Majority of demand supply; much higher than the baseline (2017) \\*
 & Automobility (private vehicles) & Lower than the baseline (2017) & Still relevant, but much lower than the baseline (2017) \\*
 & Slow modes (walking and cycling) & Moderate increase compared to baseline (2017) & Higher than the baseline (2017) \\ \addlinespace
\textit{Cultural perception} & Mobility & Accessibility as a focus, managed, reasonable travel time, integrated & Accessibility, local in scale, slowed down, managed, reasonable travel time and reliability, integrated \\*
 & Public transport & Public mobility as an affordable and accessible service & Public mobility as a reliable, comfortable, enjoyable and accessible service \\*
 & Automobility & Automobility fills accessibility gaps; symbolic status decreasing & Automobility as a utility to serve a special need \\
\textit{Public transport} & Reliability & Medium (higher than the baseline) & High \\*
 & Consumer cost & Low & Low \\*
 & Accessibility & Medium-high & High \\*
 & Safety & High & High \\*
 & Public transport infrastructure investments & High and continuous & High and continuous \\ \addlinespace
\textit{Automobility} & Reliability & High & High \\*
 & Consumer cost & Medium & High \\*
 & Accessibility & High & High (especially in rural or remote areas) \\*
 & Safety & Medium (especially Low Income Countries) & High (higher risk than public transport) \\*
 & Automobility infrastructure investments (roads, fuel stations, etc.) & Medium; maintenance dominates in HICs; capacity increased in LICs & Low to medium; maintenance covers the majority of the investments; capacity is not increased \\ \addlinespace
\textit{Fuel technology} & Automobiles & Battery electric vehicles and hybrids for short-medium ranged trips; biofuelled for long range & Battery electric vehicles for short-medium ranged trips; hydrogen fuelled for long range \\*
 & Rail & Full electrification of the network & Full electrification of the network \\*
 & Bus & Hybrid, or biofuelled & Electric or hydrogen-fuelled \\*
 & Aviation & Renewable biofuels & Renewable biofuels
\end{longtable}
}
\end{landscape}

\subsection{The backcasted path to SSP1-MOB}
\label{ss:results:backcasting-the-path}
\todoparagraph{Quick description (table? itemize?) of the changes occurring between the 2050 and 2100 narratives.}
\todoparagraph{Quick description (table? itemize?) of the changes occurring between the 2050 narrative and 2017 baseline.}
\todoparagraph{Explain and contextualise all the backcasted changes: what is the main path to the sustainable future mobility paradigm?}
As already highlighted in the \sref{ss:results:ssp1-mob-paradigm}, one of the key preconditions for the transition to a sustainable mobility paradigm is a change in the \emph{cultural perception} of mobility (how is it understood). This shift of cultural perspective happens not only at the user level, but also at the urban/traffic planner one. \textcite{banister2008_sustainablemobilityparadigm}, drawing from \textcite{marshall2001_challengesustainabletransport}, presents some of the foundation stones of the change to the sustainable mobility paradigm, with which the SSP1-MOB narrative is aligned, and also provides some insight into the cultural contrasts between such a paradigm and the current mobility culture:
%
\begin{enumeratealpha}
\item Instead of speeding up traffic, slowing down traffic is the new target for both users and planners.
\item Streets are seen as a space for urban life, rather than just public spaces occupied by private automobiles only.
\item Social and environmental multicriteria analyses regarding mobility are performed in addition the already dominant economic assessments.
\item Larger travel times become acceptable, in contrast to the current ever accelerating pace of society and mobility.
\item Attention is shifted from vehicles to people: human, personal mobility is the core of the frame, opposed to just vehicle mobility.
\end{enumeratealpha}

\todonote{finish this}