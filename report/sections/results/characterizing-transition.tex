% (Geels et al, 2012)
% To deal with sustainability issues, transport studies must examine:
%    i) the interactions between technology and behaviour
%   ii) stability and change (not solely change!)
%  iii) multi-actor dynamics
%
% FOCUS areas:
%  - MLP-based (static) characterisation:
%    [lock-in patterns, landscape, regime, niche, agents]
%    · SSP1-MOB 2100
%    · AUTOLOCK 2017
%  - Transition (dynamic) characterisation:
%    · Backcasted path, taking into account the dynamics discovered in AUTOLOCK
%
% SECTION TITLE:
%  Transitions theory integration: SSP1-MOB and AUTOLOCK under a MLP approach
%
The purpose of this section is to integrate the previous results under a coherent discursive umbrella. The integration discourse should be able to cover: system dynamics analysis, static system descriptions, change accounts, stability patterns and, most importantly, a broad range of system social and technical aspects. \emph{Transitions theory} is the chosen framework to achieve this objective, given the features that have been reviewed in the  \nameref{ss:intro:transitions-theory} subsection, in the Introduction chapter. \ssref{ss:results:transition_mlp} provides an overview of (1) the desired SSP1-MOB vision and (2) the current automobility system as described by the AUTOLOCK conceptual model under the \emph{multi-level perspective} (MLP) analytical framework. Next, the dynamics of the transition from 2017's system to the 2100 SSP1-MOB vision are characterised in \ssref{ss:results:transition_dynamics}. The dynamics characterisation is based on the SSP1-MOB backcasting results.

\subsection{MLP analysis of SSP1-MOB and AUTOLOCK}
\label{ss:results:transition_mlp}
% MLP perspective on transitions:
%  - Niches: locus for radical change
%    · articulation of expectations or visions
%    · building of social networks
%    · learning processes on various dimensions
%  - Socio-technical regimes: locus for "dynamic stability" (incremental change)
%     Actors within the regime share a set of rules and practices
%  - Socio-technical landscape: the wider context that influences niche and
%    regime dynamics, including: ideologies, belief, macro-economics, etc.
%

\subsection{Transition dynamics characterisation}
\label{ss:results:transition_dynamics}
% Transition patterns:
%  - Transition pathways:
%    · Transformation
%    · De-alignment and re-alignment
%    · Technological substitution
%    · Reconfiguration pathway
% - Add-on and hybridisation
% - Knock-on effects and innovation cascades
% - Fit-stretch pattern
% - Hype-disappointment cycles
% - Niche-accumulation pattern
%

%
% ******************************************************************************
%
\subsection*{THE FOLLOWING ARE JUST FRAGMENTS TO BE SORTED OUT}
\todoparagraph{SSP1-MOB What are the regimes, niches and landscape features? What about transitions? \textbf{only static elements such as regimes. Not dynamics!! That is for the backcasting process}}

On a broader (landscape) scale, cultural discourses have to change to support collaborative and collective efforts instead of increased (and alienating) individualisation. In order to move away from ever growing congestion and systemic speed-up, the culture of acceleration that stems from modern capitalism must be challenged and reverted or put to a halt \textcite{zijlstra2012_SocioSpatialPerspective}.

Because of the enormous potential of inertia that infrastructural and urban patterns pose with respect to achieve the necessary paradigmatic transition of mobility, political commitment to change the way cities and landscapes are planned is urgent.

The changing urban form has to be backed by a public transport infrastructure like the one described in SSP1-MOB; one that supports massive personal transportation at low environmental, social and economic costs. Other mobility-related aspects found in SSP1-MOB, that make it differ from the current situation are the facilities that must be provided for slow modes like cycling to become feasible and desirable.

\todowarning{Where should this be?}
Travelling is discouraged by governments through economic policy instruments such as carbon and/or congestion taxes, road usage fees and by shifting investments and subsidies from private mobility schemes to public ones. Fuel subsidies have progressively been reduced or abandoned, road building has come to a halt, in favour of enhanced public transport infrastructure (rail networks, bus lanes, etc.) and public transport fees are now heavily subsidized, instead of investing these resources in automobility fleet renewal programmes or in facilitating the operation of the automobility industry.

Regarding aviation, economic incentives for low-cost trips have been reduced; the market has been regulated and stringent regulations on emissions and other indirect impacts have been put in place.