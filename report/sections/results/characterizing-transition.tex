% (Geels et al, 2012)
% To deal with sustainability issues, transport studies must examine:
%    i) the interactions between technology and behaviour
%   ii) stability and change (not solely change!)
%  iii) multi-actor dynamics
%
% FOCUS areas:
%  - MLP-based (static) characterisation:
%    [lock-in patterns, landscape, regime, niche, agents]
%    · SSP1-MOB 2100
%    · AUTOLOCK 2017
%  - Transition (dynamic) characterisation:
%    · Backcasted path, taking into account the dynamics discovered in AUTOLOCK
%
% SECTION TITLE:
%  Transitions theory integration: SSP1-MOB and AUTOLOCK under a MLP approach
%
The purpose of this section is to integrate the previous results under a coherent discursive umbrella. The integration discourse should be able to cover: system dynamics analysis, static system descriptions, change accounts, stability patterns and, most importantly, a broad range of social and technical aspects. \emph{Transitions theory} is the chosen framework to achieve this objective. \ssref{ss:results:transition_mlp} provides an assessment of the transition from the current automobility system, as described by the AUTOLOCK conceptual model, to the desired SSP1-MOB vision, under the \emph{multi-level perspective} (MLP) analytical framework.%TODO IF THE LAST SECTION IS ADDED, UNCOMMENT: Next, the ``dynamics'' of this transition are characterised in \ssref{ss:results:transition_dynamics}.

\subsection{From AUTOLOCK to SSP1-MOB: MLP analysis}
\label{ss:results:transition_mlp}
% MLP perspective on transitions:
%  - Niches: locus for radical change
%    · articulation of expectations or visions
%    · building of social networks
%    · learning processes on various dimensions
%  - Socio-technical regimes: locus for "dynamic stability" (incremental change)
%     Actors within the regime share a set of rules and practices
%  - Socio-technical landscape: the wider context that influences niche and
%    regime dynamics, including: ideologies, belief, macro-economics, etc.
%d
The following analysis of the transition using the multi-level perspective does not assess the current situation (represented by the AUTOLOCK model) and the desired future (described in the SSP1-MOB narrative) separately. Instead, an overview of the transition is given in a prospective fashion. All along the process of this transition, the dominance of regimes, the landscape pressures and the innovations in the niches evolve; this section aims to explain the co-evolution of these elements.

\subsubsection*{The 2017 automobility socio-technical regime and the landscape that supports it}
In the current mobility system, automobility is, by far, the dominant socio-technical regime. As reviewed within the AUTOLOCK model discussion, automobility has been able to not only dominate mobility, from the travel demand point of view, but has also shaped the lifestyles of generations, driving an adaptation of cultural practices to the spaces and requirements of this form of mobility. The automotive industry, politicians, researchers, planners and the very civil society have embraced this mobility paradigm to the point where alternative, cost-effective and efficient travel modes such as public transport or walking and cycling have been relegated to perform a background and marginal role. Urban landscapes have changed in the last century to accommodate automobiles, with huge road networks being constructed and shifting the focus of urban planning to a car-centric approach. Commuting to work and school, shopping, family visits and recreational activities have all been built around the central role of the automobile.

This pattern of infrastructure development, together with the vested interests shown by the automotive industry and the sunk investments of governments around the globe to support automobility pose an enormous inertia that resists radical change. It is true that technical developments and efficiency increases have been brought to the automobile world: engine improvements, enhanced safety, etc. However, the fundamentals of the system remain. Cars may change in shape and internally, but they serve the same ultimate purpose of personal, private mobility; it is a form of ``dynamic stability''. The system is configured in such a manner that all the elements that support it are inter-locked. The automobility culture/paradigm cannot be understood without the physical infrastructure to drive around, nor without the extensive network of refuelling stations and mechanic workshops, nor without a pervasive imposed\todonote{Find a better word?}~notion that automobiles are the best alternative for mobility, because they liberate the individual from the restraints of society.

As revealed in the AUTOLOCK model, the automobility regime is deeply entrenched in the cultural background of our society. Not only through the changes in cultural mobility practices, but as a symbol of the ruling (neo)liberal ideology. Unconstrained individual freedom discourses find in the automobile the perfect mean for liberation, for consumption and for reification from a utility to a meaning that goes beyond that of pure mobility. It is a symbol of status, of positive personal success (in the terms of materialistic capitalism), of the victory of the self over the anonymous mass, of a unique and distinct identity. The legitimacy of automobility can hardly be contested from environmental or social concerns: culture and ideology prevail.

Therefore, the automobility regime has become resilient to changes and pressures from the landscape and innovation niches so far. It relies on the social and physical complex that supports it on one side (infrastructure, investments and industries), and on the apparent legitimacy of the culture it has helped to shape. A transition to a sustainable form of mobility that dismisses automobility and embraces other travelling modes will have to counter these forces. Moreover, the transition will require efforts from all sorts of agents (stakeholders), ranging from policy makers at the top most spheres to individuals that mould the culture of mobility with their choices and behaviours.

There are many aspects of the discussed automobility regime that rely on trends at the landscape level to be legitimised. A lot has been talked about the liberal ideology and its influence in the development of automobility throughout this thesis. However, there are more influential facets to the ideology than just the ideal (value) of individual freedom. Maximisation of \emph{private property} (capital) is another central tenet of the neoliberal economy that supports the car regime. In contrast, the fact that public transport is not owned by the traveller renders it as a ``lower'' alternative to opt for. The same applies to walking, where no private nor public property at all is necessary. The fact that cars are private, when embedded in a framework of values that primes private property makes them appealing to the individuals \parencite{zijlstra2012_SocioSpatialPerspective}.

Arguably, another feature that can be attributed to the landscape that supports automobility is the energetic dependence on hydrocarbons. The reliance of the current global energy system on fossil fuels is huge. Their condition as cheap and dense energy carriers causes that, despite minor (or major) crises, fuel prices are low. Low fuel prices drive automobility further ahead, acting as another reinforcing mechanism \parencite{wells2012_NatureCausesInertia}. Until other energy carriers become available at reasonable prices, internal combustion engines will dominate the technological aspect of the automobility regime, polluting and threatening human and ecological health.

\subsubsection*{Automobility challenged: niches of innovation and landscape pressures}
The previous paragraphs paint a picture of automobility that is discouraging, from the sustainability point of view. It seems that automobiles cannot be taken out of the streets, due to the reinforcing mechanisms that stabilise it against all threats. However, this is not the complete picture. Automobility is not completely unchallenged any more. Innovation niches in mobility have emerged. Technology-wise, there have not been true breakthroughs that open the possibility to liberate us from automobiles, but some transport technologies have been rethought into innovative ways to fight for a reduction in car use. For example, municipal bike rental services have emerged in many cities around the world, offering more than 2 million public use bikes in 2016, doubling the figures of 2015 \parencite{meddin2017_BikesharingWorld}.

Besides technological options, another type of niches that challenge the dominance of the automobility regime are what \textcite{zijlstra2012_SocioSpatialPerspective} call \emph{socio-spatial niches}. \emph{Slow cities}, spatial policies that discourage automobile use in favour of other modes (promotion of bike and bus lanes, restriction of parking spaces, etc.) and sustainable urban planning approaches (new schools of urban planning practices) fall under this niche category. \CLDvar{Mixed-use urban development}, together with \CLDvar{Demand management policies} are examples of such niches that the AUTOLOCK model incorporates. Real world examples can be found of sustainable urban planning practices; it is the case of the Spanish city of Pontevedra, in the region of Galicia, where a mere 28\% of travel mode share is attributed to automobiles, there have been no fatal accidents in over three years (in a city of \num{65000} inhabitants), 80\% of the children walk to school and more than \SI{40}{\kilo\meter} of pedestrian/cycling paths have been built since 1999 \parencite{precedo2017_Pontevedraelsueno}.

Some other niches appear in the form of contra-cultures to the established automobile-based lifestyle described in AUTOLOCK --- they are not explicit in the model, but they conform the potential balancing forces in the dynamics analysed in \ssref{ss:results:cld_cultural-feedbacks}. Alternative mobilities cultures, such as bicycling, or low-mobility lifestyles are an example of these. More important than the actual mobility practices of these contra-cultural movements are the discourses that support them and at that challenge the dominant regime. Anti-consumerism, sustainability advocacy, public health defence, environmentalism and social cohesion fights for livable and equitable cities can be counted among these discourses. An underlying assumption of the SSP1-MOB narrative is that these niche discourses will gain momentum until they are promoted to the landscape level.

Beyond niches, there are some signs that pressure is building up at the landscape level to constrain the dominance of automobility as a regime. \ce{CO_2} emission caps force governments to increase the pressure for de-carbonising the transport sector. Despite previous efforts by governments to amend this situation through efficiency improvements (which are clearly insufficient\todocite{Some data here?}), the recent Paris agreement to combat climate change may be a beacon of hope to force more stringent regulations on \ce{CO_2} emissions from this sector. Additionally, the very same dependence on fossil fuels that has kept automobiles on the roads unchallenged for decades may well be a force for change. With oil reserves running low and extraction costs rising (the observed effects of \emph{peak oil}), the steady supply of fuel that automobiles require may be at risk. In a world with ever-decreasing fuel availability, pressure is and will be growing in order to abandon this oil thirsty transport mode. This pressure is exemplified by the introduction of battery electric vehicles (a technological niche, still) capable of long driving ranges at ``normal'' automobile speeds. However, it must be noted that electric cars are not a solution to automobile regime problems such as congestion, accidents or unsocial spatial organisation (single-use urban developments, separation of work and living spaces, increased commuting time, etc.).

\subsubsection*{Mobility intermediate futures: landscape and regime reconfigurations in 2050}
Down the path of the transition to SSP1-MOB (around the year 2050), the landscape is assumed to change substantially. This is combined with the pressure exerted by technological and cultural niches to challenge the regime of automobility, driving a progressive but certain transition to a more sustainable form of mobility. At the landscape level, (a) sustainability concerns among the population have grown considerably, (b) population itself has grown, but slowly, (c) economic growth has been relatively high, (d) technology has developed rapidly, enabling efficiency increases, (e) material consumption has shrunk and (f) governance systems are effective at the national and international levels \parencite{vuuren2017_Energylanduse}. This development pathway trickles down onto several mobility-related trend shifts, ranging from energy/power/fuel system reconfigurations to changes in the urban spatial organisation.

On the side of land use patterns (sometimes considered as part of the landscape of mobility; others an integral part of the system or regime), urban density is assumed to grow, allowing for a shrinkage of travel distances. This alone already puts pressure on the automobility regime, since the use of cars for the daily life is no more an imperative --- it is a balancing force that breaks the \CLDloop{R3} \emph{Automobility dependence} reinforcing loop in AUTOLOCK. Moreover, due to the combination of population growth and high concerns for sustainability issues, the urban planning paradigm has moved towards mixed-use developments. While it is not yet a complete transition --- (a) it is not the norm yet and (b) already developed (sub)urban areas take a long time and resources to renovate ---, it is certainly not a marginal (niche) practice any more: it has surpassed the local/municipal level and has permeated the regional planning levels. This is a second and very important trend that breaks another lock-in mechanism of the automobility regime, because social facilities and living spaces stop being separated far away from each other (implementing the \CLDloop{B3} balancing loop in AUTOLOCK). Automobility starts to not make sense, now that everything is within walking or public transport reach.

The changing urban form is backed by an increasingly extensive public transport network. Railways are being electrified globally, thus reducing operating costs, and high-speed train lines are being built, displacing, once again, the automobile from the long range trips ``market'' \parencite{vuuren2017_Energylanduse}. Intermodal travel, despite not being fully utilised, is facilitated by the provision of integrated travel information through IT services. Bus networks have also grown in size, now that rural area residents shift from car to public transport use. This new distribution of mode shares is the indication of a key aspect of the transition to SSP1-MOB. Even though automobility still dominates in terms of travel demand over any other mode, the aggregated demand attributed to all public transport modes is on par to that of cars. Instead of a single socio-technical transport system prevailing over others (automobility in 2017), the observed trend in 2050 is that public transport and automobility are now co-existent as full-fledged regimes. The reasons why both can be understood as regimes are: (1) actors within the systems share a common set of practices and rules (particular to each system), (2) they evolve in a form of dynamic stability and (3) they both dominate over radically innovative niches, such as bicycling-only mobilities or other imaginative mobility technologies\footnote{It is worth noting that the discussion of particular niches in the future is somehow futile. Regimes tend to endure and only change over the span of decades, but niches rise and fall rapidly --- they are volatile. Besides this, innovative technological niches are hard to predict, because of their uncertain nature.}.

Fuel technologies, also in between the landscape and regimes level, have also evolved before 2050. On one hand, fossil fuels still account for a big chunk of the total (global) energy demand --- certainly oil, but coal and natural gas too, as dense energy sources for electricity generation ---, but bio-fuels are starting to gain share; on the other hand, the power grid has grown in size and has reduced costs, easing the electrification of the transport sector \parencite{vuuren2017_Energylanduse}. Cheap and available electricity, coupled with a rise in the share of bio-fuels indicates that the reinforcing dynamics of cheap fuel and oil-dependence within automobility are also starting to be balanced out. The increasing demand for electric vehicles, especially in road and rail due to their perceived cleanliness, accentuates this feedback rupture.

Finally, mobility cultures are in the process of a transition too. There is less support for the automobility regime now, because the reasons that legitimised its dominance are becoming weaker. At the landscape level, the imperative of unfettered individual freedom promoted by the neoliberal ideology has been challenged globally, for two reasons: (1) the realisation that individualist-consumerist behaviours cause deeply negative impacts in the environment (in accordance to the assumed sustainability concerns trend of SSP1/SSP1-MOB) and (2) the economic and social inequality that are generated by the neoliberal policies worldwide. Another landscape trend with respect to culture is the rise in sustainable consumption patterns: self-identity is not provided by materialistic artefacts so much as in 2017, thus stripping car ownership of part of its symbolism.

\subsubsection*{Completing the transition to SSP1-MOB in 2100}
The world has completed (or advanced in) a series of changes by the year 2100, which can be categorised as landscape developments. The energy system has finally been (mostly) de-carbonised: solar, wind and hydro power are the main sources of electricity generation; bio-fuels dominate over fossil ones and hydrogen fuel cell technology (and infrastructure) has been widely made available for road vehicles. Urban densities are high and cities have become compact nuclei of mixed-use developments: the need for rapid and long-distance mobility has greatly been reduced. Single-use planning is a thing of the past and a de-materialised economy also means that lesser transport is needed. Climate change and other sustainability issues have been at the core of national and international policy, because the population has grown highly concerned about it. An enormous pressure from the ``bottom'', in the form of various socio-technical niches, has caused a cascade of cultural and political changes.

Public transport has the strongest political support ever and its aggregated demand is superior to that of automobiles, positively confirming its position as the dominant socio-technical regime. The rail system has completely been electrified, high-speed trains are common and are heavily used and buses have become electric or fuelled by hydrogen. Intermodal travel is very much facilitated and is commonly used, due to a higher degree of acceptance (especially with regards to the relatively longer travel times) and convenience. Accessibility through public transport is on par with that of the automobile, regarding the public option as more than appealing, from an economic and cultural point of view --- it is important to stress the shift towards a sustainability concerned lifestyle that people around the globe has embraced. In this respect, the landscape provides a feature that legitimises public transport over automobility (whenever it is more feasible and convenient): the \emph{common good} is the moral imperative, rather than \emph{individual freedom}.

Automobility has, on the other hand, fallen in terms of demand, but still constitutes a regime, in the sense that the complexity, evolution patterns and interactions between actors in the system follow the same principles of the regime it had been. Therefore, the future of SSP1-MOB has been definitely established as a multi-modal mobility paradigm. Public transport dominates as a regime, but automobility still fills the gaps in accessibility, convenience or economic efficiency that public transit cannot cover. With respect to the environmental sustainability of the automobility system, the main technologies that drives most cars now are sophisticated electric batteries and hydrogen fuel cells --- \ce{CO_2} emissions have therefore been cut and dependence on fossil fuels has dropped considerably. New reinforcing mechanisms have risen, inter-locking automobility with the electricity system, for example, while old ones are broken, such as the inter-dependency of cars and single-use (sub)urban developments.

Finally, the culture of mobility of the SSP1-MOB world has transitioned to a more humanist one. Not only the symbolic nature of automobiles has been completely dismantled (they are regarded as utilities), but integration, management and \emph{reasonable} travel times (vs. minimised times) are the new central focus of mobility. The old legitimacy of the cars regime, based on preceding ideological values, has given way and public transport is now as legit as automobiles. Despite the higher figures of public transport demand, walking and cycling have also re-gained their fundamental role in personal mobility, especially in urban areas. They are used as mediators for faster modes connection (inter-modal travel) and as the basic method to reach shopping, work, school and family locations. Despite their technical simplicity, the (finally) shared practices and rules --- between planners, builders, pedestrians, cyclists and citizens in general --- that enable these slow modes, along with the need for the supporting infrastructure, elevate them to the level of a socio-technical regime. This reinforces the notion that mobility has effectively become multi-modal in nature. Furthermore, mobility is now environmentally sustainable and socially human.

%TODO                                              FOLLOWING SECTION IS OPTIONAL
%\subsection{From AUTOLOCK to SSP1-MOB: transition dynamics}
%\label{ss:results:transition_dynamics}
%% Transition patterns:
%%  - Transition pathways:
%%    · Transformation
%%    · De-alignment and re-alignment
%%    · Technological substitution
%%    · Reconfiguration pathway
%% - Add-on and hybridisation
%% - Knock-on effects and innovation cascades
%% - Fit-stretch pattern
%% - Hype-disappointment cycles
%% - Niche-accumulation pattern
%%
%\todoparagraph{Describe the transition in terms of the patterns observed:\\
%-- Transition pathways: transformation vs de-alignment and realignment vs technological substitution vs reconfiguration pathway\\
%-- Add-on and hybridisation\\
%-- Knock-on effects and innovation cascades\\
%-- Fit-stretch pattern\\
%-- Hype-disappointment cycles\\
%-- Niche-accumulation pattern
%}
