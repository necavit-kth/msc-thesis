% (Geels et al, 2012)
% To deal with sustainability issues, transport studies must examine:
%    i) the interactions between technology and behaviour
%   ii) stability and change (not solely change!)
%  iii) multi-actor dynamics
%
% FOCUS areas:
%  - MLP-based (static) characterisation:
%    [lock-in patterns, landscape, regime, niche, agents]
%    · SSP1-MOB 2100
%    · AUTOLOCK 2017
%  - Transition (dynamic) characterisation:
%    · Backcasted path, taking into account the dynamics discovered in AUTOLOCK
%
% SECTION TITLE:
%  Transitions theory integration: SSP1-MOB and AUTOLOCK under a MLP approach
%
The purpose of this section is to integrate the previous results under a coherent discursive umbrella. The integration discourse should be able to cover: system dynamics analysis, static system descriptions, change accounts, stability patterns and, most importantly, a broad range of social and technical aspects. \emph{Transitions theory} is the chosen framework to achieve this objective. \ssref{ss:results:transition_mlp} provides an assessment of the transition from the current automobility system, as described by the AUTOLOCK conceptual model, to the desired SSP1-MOB vision, under the \emph{multi-level perspective} (MLP) analytical framework. Next, the ``dynamics'' of this transition are characterised in \ssref{ss:results:transition_dynamics}.

\subsection{MLP analysis of SSP1-MOB and AUTOLOCK}
\label{ss:results:transition_mlp}
% MLP perspective on transitions:
%  - Niches: locus for radical change
%    · articulation of expectations or visions
%    · building of social networks
%    · learning processes on various dimensions
%  - Socio-technical regimes: locus for "dynamic stability" (incremental change)
%     Actors within the regime share a set of rules and practices
%  - Socio-technical landscape: the wider context that influences niche and
%    regime dynamics, including: ideologies, belief, macro-economics, etc.
%
\todonote{Not true anymore! Re-write! Now: the MLP discussion is based on the transition process itself: current situation (regime + niches), changing landscape and, finally, new regime configuration in 2100.}The analysis of the transition using the MLP is divided into each of the levels of the analytical framework: the socio-technical regime, the socio-technical landscape and the innovation niches.

\subsubsection*{The automobility socio-technical regime}
In the current mobility system, automobility is, by far, the dominant socio-technical regime. As reviewed within the AUTOLOCK model discussion, automobility has been able to not only dominate from the travel demand point of view, but has also shaped the lifestyles of generations, driving an adaptation of cultural practices to the spaces and requirements of this form of mobility. Urban landscapes have changed in the last century to accommodate automobiles, with huge road networks being constructed and shifting the focus of urban planning to a car-centric approach. The automotive industry, politicians, researchers, planners and the very civil society have embraced this mobility paradigm to the point where alternative, cost-effective and efficient travel modes such as public transport or walking and cycling have been relegated to perform a background and marginal role. Commuting to work and school, shopping, family visits and recreational activities have all been built around the central role of the automobile.

As revealed in the AUTOLOCK model, the automobility regime is deeply entrenched in the cultural background of our society. Not only through the % changes in cultural mobility practices, but as a symbol of the ``imperant'' (neo)liberal ideology. Freedom and all that

\subsubsection*{Automobility challenged: niches of innovation and landscape pressures}
...but it is not completely unchallenged any more. Niches have emerged. There are many counter-cultural movements that challenge car dominance. There is increasing acknowledgement in society of the negative impacts of cars. Electric vehicles have come to solve some of the problems. But people are going further ahead: car free cities like Pontevedra prove that it is possible.

CO2 emmissions, dependance on fossil fuels. Paris Climate agreement. Da di da

\subsubsection*{Mobility futures: regime re-configuration}
SSP1-MOB mobility future is based on a different regime configuration: one in which multi-mode mobility dominates. Support for public transport rises to make it dominate over other forms of mobility, but walking and cycling also re-gain their fundamental role in personal mobility, especially in urban areas. They are used as mediators for faster modes connection (inter-modal travel) and as the basic method to reach shopping, work, school and family locations.

Mobility culture 

\subsubsection*{SSP1-MOB socio-technical landscape}
Discourses on freedom do not prevail over the common good. Struggles to fight climate change have pushed a new era of humanism, collaboration and international cooperation. 


\subsubsection*{Niches}
\todonote{Adapt this to the new narrative (following the transition)}Although not being the emphasis of the AUTOLOCK model in \sref{s:results:autolock-model}, several niches co-exist with the dominant automobility regime nowadays. It must be noted that public transportation, despite being relegated to a ``second position'' in most countries (and being certainly marginal in some others)\todocite{Cite some statistics of car vs PT demand}, is actually not considered a niche, because it does not present patterns of radical innovation nor it is articulated through visions and expectations, both characteristic feature of niches \parencite{geels2012_MultiLevelPerspective}. This holds true for the SSP1-MOB future as well, because public transport plays, in this case, the role of the dominant regime, or rather, that of a travel mode that co-exists with others, such as slow modes, in a multi-modal regime.

One type of niches that challenge the dominance of the automobility regime are what \textcite{zijlstra2012_SocioSpatialPerspective} call \emph{socio-spatial niches}. \emph{Slow cities}, spatial policies that discourage automobile use in favour of other modes (promotion of bike and bus lanes, restriction of parking spaces, etc.) and sustainable urban planning approaches (new schools of urban planning practices) fall under this niche category. \CLDvar{Mixed-use urban development}, together with \CLDvar{Demand management policies} are examples of such niches that the AUTOLOCK model incorporates. However, it was already discussed in \ssref{ss:results:cld_urban-planning} that the reinforcing dynamics of the regime hinder the further development of those niches --- unless there is a strong political will to revert the situation. In the world of SSP1-MOB, it is assumed that there is such a will to embrace the practices of this socio-spatial niches and elevate them to the level of a dominant regime. This is, in the future of SSP1-MOB it is car-centric urban developments that become niches\footnote{Automobile-centric developments are not radical innovations in 2100, therefore not qualifying as \emph{niches} in the sense that \textcite{geels2012_MultiLevelPerspective} gives it. Rather, they would fill a less important, more specialised ``market'' space, if there was the need for it.}.

Some other niches appear in the form of contra-cultures to the established automobile-based lifestyle in AUTOLOCK --- again, they are not explicit in the model, but they conform the potential balancing forces in the dynamics analysed in \ssref{ss:results:cld_cultural-feedbacks}. Alternative mobilities cultures, such as bicycling, or low-mobility lifestyles are an example of these. More important than the actual mobility practices of these contra-cultural movements are the discourses that support them and at that challenge the dominant regime. Anti-consumerism, sustainability advocacy, public health defence, environmentalism and social cohesion fights for livable and equitable cities can be counted among these discourses.

\subsection{Transition dynamics characterisation}
\label{ss:results:transition_dynamics}
% Transition patterns:
%  - Transition pathways:
%    · Transformation
%    · De-alignment and re-alignment
%    · Technological substitution
%    · Reconfiguration pathway
% - Add-on and hybridisation
% - Knock-on effects and innovation cascades
% - Fit-stretch pattern
% - Hype-disappointment cycles
% - Niche-accumulation pattern
%
\todoparagraph{Describe the transition in terms of the patterns observed:\\
-- Transition pathways: transformation vs de-alignment and realignment vs technological substitution vs reconfiguration pathway\\
-- Add-on and hybridisation\\
-- Knock-on effects and innovation cascades\\
-- Fit-stretch pattern\\
-- Hype-disappointment cycles\\
-- Niche-accumulation pattern
}
%
% ******************************************************************************
%

\todowarning{\textbf{THE FOLLOWING ARE JUST FRAGMENTS TO BE SORTED OUT}}

On a broader (landscape) scale, cultural discourses have to change to support collaborative and collective efforts instead of increased (and alienating) individualisation. In order to move away from ever growing congestion and systemic speed-up, the culture of acceleration that stems from modern capitalism must be challenged and reverted or put to a halt \textcite{zijlstra2012_SocioSpatialPerspective}.

Because of the enormous potential of inertia that infrastructural and urban patterns pose with respect to achieve the necessary paradigmatic transition of mobility, political commitment to change the way cities and landscapes are planned is urgent.

The changing urban form has to be backed by a public transport infrastructure like the one described in SSP1-MOB; one that supports massive personal transportation at low environmental, social and economic costs. Other mobility-related aspects found in SSP1-MOB, that make it differ from the current situation are the facilities that must be provided for slow modes like cycling to become feasible and desirable.

\todowarning{Where should this be?}
Travelling is discouraged by governments through economic policy instruments such as carbon and/or congestion taxes, road usage fees and by shifting investments and subsidies from private mobility schemes to public ones. Fuel subsidies have progressively been reduced or abandoned, road building has come to a halt, in favour of enhanced public transport infrastructure (rail networks, bus lanes, etc.) and public transport fees are now heavily subsidized, instead of investing these resources in automobility fleet renewal programmes or in facilitating the operation of the automobility industry.

Regarding aviation, economic incentives for low-cost trips have been reduced; the market has been regulated and stringent regulations on emissions and other indirect impacts have been put in place.