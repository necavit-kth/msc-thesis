\todoparagraph{SSP1-MOB What are the regimes, niches and landscape features? What about transitions? \textbf{only static elements such as regimes. Not dynamics!! That is for the backcasting process}}

\todoparagraph{BASELINE What are the regimes, niches and landscape features? What about transitions?}

On a broader (landscape) scale, cultural discourses have to change to support collaborative and collective efforts instead of increased (and alienating) individualisation. In order to move away from ever growing congestion and systemic speed-up, the culture of acceleration that stems from modern capitalism must be challenged and reverted or put to a halt \textcite{zijlstra2012_SocioSpatialPerspective}.

Because of the enormous potential of inertia that infrastructural and urban patterns pose with respect to achieve the necessary paradigmatic transition of mobility, political commitment to change the way cities and landscapes are planned is urgent.

\todowarning{Where should this be?}
Travelling is discouraged by governments through economic policy instruments such as carbon and/or congestion taxes, road usage fees and by shifting investments and subsidies from private mobility schemes to public ones. Fuel subsidies have progressively been reduced or abandoned, road building has come to a halt, in favour of enhanced public transport infrastructure (rail networks, bus lanes, etc.) and public transport fees are now heavily subsidized, instead of investing these resources in automobility fleet renewal programmes or in facilitating the operation of the automobility industry.

Regarding aviation, economic incentives for low-cost trips have been reduced; the market has been regulated and stringent regulations on emissions and other indirect impacts have been put in place.