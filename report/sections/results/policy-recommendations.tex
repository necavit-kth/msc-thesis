% (Geels et al, 2012)
% The logic of the MLP suggests policy makers can follow two strategies
% to influence transitions:
%   i) Enhance pressure on regimes through economic instruments and regulation
%  ii) Stimulate the emergence and diffusion of niche innovations
\todoparagraph{The logic of the MLP suggests policy makers can follow two strategies to influence transitions:\\
i) Enhance pressure on regimes through economic instruments and regulation\\
ii) Stimulate the emergence and diffusion of niche innovations
}

%TODO            PUT THE FOLLOWING SOMEWHERE!!
Travelling is discouraged by governments through economic policy instruments such as carbon and/or congestion taxes, road usage fees and by shifting investments and subsidies from private mobility schemes to public ones. Fuel subsidies have progressively been reduced or abandoned, road building has come to a halt, in favour of enhanced public transport infrastructure (rail networks, bus lanes, etc.) and public transport fees are now heavily subsidized, instead of investing these resources in automobility fleet renewal programmes or in facilitating the operation of the automobility industry.

Regarding aviation, economic incentives for low-cost trips have been reduced; the market has been regulated and stringent regulations on emissions and other indirect impacts have been put in place.

\subsection{Niche level policies}
\label{ss:results:policies_niche}
\todoparagraph{Encourage niche formation with two goals: (1) provide funding for potential breakthrough technological changes and (2) promote counter-cultural movements that challenge the regime}

\todoparagraph{Niche formation is difficult to predict, therefore specific policies are absurd to describe in here.}

\subsection{Regime level policies}
\label{ss:results:policies_regime}
\todoparagraph{Policies: (1) finish the ``good relationship'' with the automotive industry, (2) stop subsidising fuels, (3) stop subsidising car fleet renewal schemes, (4) introduce congestion taxes in urban areas, (5) change urban development patterns, through regulation and changes in the education system (for planners in the making), (6) more stringent regulation on emmissions, (7) invest in public transportation development: accessibility, reliability, frequency, much higher subsidies, etc.}

\subsection{Landscape level policies}
\label{ss:results:policies_landscape}
\todoparagraph{Targeted at higher levels of policy-making? National and supra-national, perhaps}
\todoparagraph{Carbon taxes. Import taxes for automobiles. Aviation regulation. International high-speed railway networks. Electric grids integration. Renewable energy (electricity) generation. Fossil fuel phase-out. Push for biofuels: regulation? International cooperation and collaboration; technological transfer. Integrated policies across countries.}