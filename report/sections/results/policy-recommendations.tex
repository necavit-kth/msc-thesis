% (Geels et al, 2012)
% The logic of the MLP suggests policy makers can follow two strategies
% to influence transitions:
%   i) Enhance pressure on regimes through economic instruments and regulation
%  ii) Stimulate the emergence and diffusion of niche innovations
This final section of the Results chapter covers the ultimate goal of the thesis: deriving \gls{PR} for a transition to a sustainable mobility future\footnote{Some of the recommendations are more indeterminate than others, due to the uncertainties and broad spectrum of the policy targets, e.g., niche promotion policies (specific niches are unexpected).}. A transitions oriented approach backs the suggestions, which are categorised into \emph{niche} (\ssref{ss:results:policies_niche}), \emph{regime} (\ssref{ss:results:policies_regime}) and \emph{landscape} level policies (\ssref{ss:results:policies_landscape}). Target audiences for the \glspl{PR} are given too, since some of them require international governance cooperation and some others are suited to local administrations. Finally, the transitions theory MLP approach suggests two general directions that policy makers can follow to induce the required changes for a transition: (a) put pressure on the regimes via legislation, regulation and economic instruments and (b) stimulate socio-technical innovations at the niche level \parencite{kemp2007_Transitionmanagementas,geels2012_AutomobilityTransitionSocio}.

\subsection{Niche level policies}
\label{ss:results:policies_niche}
Niche level policies are based on the assumption that forecasting the emergence or success of socio-technical niches to overthrow regimes is difficult or even impossible. However, following the scholarship of \emph{transition management} studies, policies can be put in place to, for example, ensure that already existent niches are guaranteed a safe space to develop \parencite{kemp2011_TransitionManagementas}. To do so, protection could be provided until the socio-technical niches gain enough momentum to compete against the regime (through competition agencies at the national or international level). Local, regional, national or international agents such as governments or private initiatives should also guarantee enough funding, through investment programmes, so that niches can sustain their radical innovation process.

With regards to the financing option, not only direct investments can be granted to specific actors within a niche: research initiatives at universities or institutes or even at private companies should be reinforced, especially those that pose to be promising in debunking existing regimes. This way, already in-place niches can be further developed and new niches can emerge. Special care should be taken for niches that go beyond technological improvements (such as battery electric vehicles). Initiatives that build counter-cultural movements (contrary to the established unsustainable mobility culture) should be promoted by the authorities. At the same time, these institutional agents should set a common vision for the kind of sustainable mobility they desire (across governance levels) and, most importantly, have the will to follow it and definitively distance themselves from the current regime.

One last comment is worth making with regards to niche-supporting policy: there is a strong need for policy makers to research the niches eligible of support and to understand the mechanisms of innovation that drive them. Furthermore, there is an urgent necessity of policy coordination and of a disassociation from opportunistic and fragmented policy programmes that serve political agendas\footnote{\emph{Political} is used here in reference to \emph{politics}, which differs from \emph{policy}. Politics deals with power struggles in government, while policy is the consistent set of actions and methods that orient current and future decisions.} \parencite{kemp2004_ManagingTransitionSustainable}.

\subsection{Regime level policies}
\label{ss:results:policies_regime}
Considering SSP1-MOB as the desired vision for mobility, a two-fold strategy is recommended by this study to policy makers in order to achieve a successful sustainable transition. First of all, it is clear that support for the current automobility regime must be dropped significantly. Secondly, public transport must be increasingly supported as the alternative regime that competes with automobility.

At the local level (municipalities), reducing car usage can be achieved with the introduction of, for example, congestion taxes\footnote{Congestion charges have been successfully applied in London and Stockholm, Gothenburg, Durham, Valletta and Milan, in Europe.}. Other subtler measures for traffic calming include the design of road networks that impede or obstruct automobile transit\footnote{The city of Pontevedra introduced such mechanisms, reaching a record low level of automobility \parencite{precedo2017_Pontevedraelsueno} and the city of Barcelona is experimenting with ``super-aisles'', where chunks of streets become isolated from regular traffic (only pedestrians, cyclists and neighbours can access the blocks), hindering the convenience of automobility \parencite{ajuntamentbarcelona2016_OmplimdeVida}.} and the decision to change urban use planning towards a more mixed use pattern that promotes lower mobility. Furthermore, local authorities should reinforce their commitment with public transport by investing even more in building a safe, reliable and convenient network.

Regional authorities carry the responsibility to manage large scale urban and infrastructural developments. Therefore, it is the most suitable locus for integration efforts of local public transport networks, as well as increased funding and subsidisation of this alternative regime. An integrated public transport network at the large scale is a fundamental requirement to advance in the transition path to SSP1-MOB --- regular and high-speed railways are part pf such a network, as are inter-city bus lines, for example. Cheap, long-distance public transport should be a central focus point of regional mobility policy. Moreover, in order to weaken automobility dominance, land use patterns at a broader scale can be adjusted to cut down on suburban sprawl, avoid inter-territorial (between municipalities) single-use areas that encourage car dependency and the like.

Finally, national governments, agencies, and international governance institutions should lead the transition to sustainable mobility by finishing, once and for all, the apparently symbiotic relationship they have with the automotive industry \parencite{wells2012_TransitionfailureUnderstanding,sturgeon2009_CrisisProtectionAutomotive}. Economic rescues should not be a political priority, if the regime is to be challenged. Moreover, fuel subsidies and car renewal (``scrappage'') programmes should also be left behind. Stringent emission regulations should be issued with increasing pressure in the emission levels themselves, in the compensatory taxes and by shortening the time frames for technological improvement. On the other side, high education curricula of infrastructure engineering or urban planning should include a strong focus on sustainability and sustainable mobility in particular, in order to accelerate the pace of the transition at the theoretical and practical levels.

\subsection{Landscape level policies}
\label{ss:results:policies_landscape}
The socio-technical landscape level seems difficult to influence through policies because of its definition as a broad social context where beliefs, ideologies, societal values, concerns, macroeconomic trends, infrastructure or media are included. However, some of these aspects can effectively be targeted by policies, perhaps at the national and international level. Carbon taxes are an example of such policies, in that they induce innovation in the automotive industry by indirectly orienting their research efforts to cleaner technologies \parencite{aghion2016_CarbonTaxesPath}. These type of taxes should be applied at the international level, or at least be coordinated in such spaces. Further examples of policy instruments to address sustainability in transport, beyond automobility, are international aviation taxes, emission trading for airline companies or levies on air-borne emissions \parencite{peeters2006_Innovationtowardstourism}.

Another form of landscape-level policy that complements economic instruments like the ones cited before is betting for an integrated, efficient and reliable (inter)national public transport network. High speed trains, either national or connecting several countries, should be put in place, due to their higher energy, land-use and social ``efficiency''. Strongly linked to transport infrastructure efforts, the aforementioned electric grid expansion (in size, but also in international integration) and the decarbonisation of the energy system are also landscape developments that would push for sustainable mobility. The policy recommendation in this case is to invest and promote investments in such areas, as well as passing regulations that force the necessary transition to renewable energy sources onto the private energy sector. Phasing-out fossil fuels is also mandatory for putting pressure on the automobility regime, as well as for transitioning to the energy system that SSP1-MOB relies on --- (economic and political) incentives should be put in place for the development of bio-fuel, electric battery and hydrogen fuel cell technologies.

Finally, with regards to the cultural and ideological aspects of the mobility landscape, a lot of work is still ahead in terms of policy. If the dominant neoliberal ideology is not challenged also from the political standpoint, it is unlikely that a fast sustainability transition takes place in the near-mid future. Unless the individualist and \emph{laisez fare} tenets held by market fundamentalists are de-legitimised, through cultural but also policy mechanisms, consumption patterns will not change, industries will keep business as usual and sustainability will not be a concern --- most of the pressing issues deal with impacts to the \emph{commons}, that are not recognised by free market fanatics \parencite{kumi2014_Canpost2015,cervantes2013_IdeologyNeoliberalismSustainable}. Therefore, if sustainability is the truly desired goal for human development, neoliberal policies and all the discourses that underpin them must be left behind in policy schemes.