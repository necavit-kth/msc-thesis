In order to develop a future vision that tells us how mobility is conceived in the year 2100, two main options are available: (a) write and justify a new vision from the ground up or (b) building upon already developed visions found throughout the literature. Both to reduce the time spent on this task and to increase the legitimacy\footnote{Given that no participatory process is carried to describe or develop the future vision and the corresponding backcasting, expert opinions, in the form of widely accepted scientific scenarios, are used.} of the result, the second option is chosen. \ssref{ss:results:ssp-scenarios-candidate} deals with the selection of the scenario that forms the basis of the future mobility vision, which is actually developed in \ssref{ss:results:ssp1-mob-development}. Finally, a discussion of the vision follows in \ssref{ss:results:ssp1-mob-paradigm}, contrasting the narrative to other results found in the literature.

\subsection{The SSP scenarios and the selected candidate}
\label{ss:results:ssp-scenarios-candidate}
The \gls{IPCC} has recently developed several scenarios that are the basis of their integrated climate change assessments \parencite{oneill2017_roadsaheadNarratives,vuuren2017_Energylanduse,fricko2017_markerquantificationShared,fujimori2017_SSP3AIMimplementation,calvin2017_SSP4worlddeepening,kriegler2017_Fossilfueleddevelopment}. These scenarios are called \glspl{SSP} and are defined on the basis of \emph{qualitative narratives} that contain all the necessary information on global trends to enable a further quantification step, using \glspl{IAM}, such as the IMAGE model \parencite{vuuren2017_Energylanduse}. Given that SSPs are not designed solely for the purpose of climate change studies, but are rather a description of world futures, they can be used in other disciplines and, particularly, in any kind of sustainability studies \parencite{oneill2017_roadsaheadNarratives}. Furthermore, the fact that the scenarios are developed in the form of a \emph{narrative} actually makes them closer to the concept of a ``vision'' than that of a traditional scenario: SSPs define final states, rather than trends or forecasts.

Among the five IPCC \gls{SSP} scenarios\footnote{The original narratives of all SSP1, SSP2, SSP3, SSP4 and SSP5 scenarios and the explanation of the assumptions taken to develop them can be found in \textcite{oneill2017_roadsaheadNarratives}.}, SSP1 ``\textit{Sustainability -- Taking the green road}'' is the one that implies a lower level of both adaptation and mitigation challenges with respect to climate change. Moreover, it is the one that is more aligned with the concept of Sustainable Development, due to its relatively high performance in all three pillars of sustainability: environmental conservation, social and economic sustainability (at least, economic \emph{growth} per capita). Therefore, it is the selected SSP to extend by covering the mobility sector, in order to perform the backcasting process that will be used in \sref{s:results:backcasting-ssp1-mob} to identify the necessary changes and development goals to reach a sustainable transport system in the future. The extension of the scenario, developed as a qualitative narrative is called \emph{SSP1-MOB}, from here on.

To help understanding the reference frame of the SSP1-MOB narrative, the original SSP1 storyline is reproduced below. It is highly suggested to read it, along with the summary table presented in~\aref{a:ssp1-summary-table}. Reprinted from Global Environmental Change, \textbf{42}, Brian C. O’Neill, Elmar Kriegler, Kristie L. Ebi, Eric Kemp-Benedict, Keywan Riahi, Dale S. Rothman, Bas J. van Ruijven, Detlef P. van Vuuren, Joern Birkmann, Kasper Kok, Marc
Levy, William Solecki, \textit{The roads ahead: Narratives for shared socioeconomic pathways describing world futures in the 21st century}, 169-180, Copyright (2017), with permission from Elsevier. License number 4104690633321.
%
\blockquote{\sffamily \textbf{SSP1 narrative}\\The world shifts gradually, but pervasively, toward a more sustainable path, emphasizing more inclusive development that respects perceived environmental boundaries. Increasing evidence of and accounting for the social, cultural, and economic costs of environmental degradation and inequality drive this shift. Management of the global commons slowly improves, facilitated by increasingly effective and persistent cooperation and collaboration of local, national, and international organizations and institutions, the private sector, and civil society. Educational and health investments accelerate the demographic transition, leading to a relatively low population. Beginning with current high-income countries, the emphasis on economic growth shifts toward a broader emphasis on human well-being, even at the expense of somewhat slower economic growth over the longer term. Driven by an increasing commitment to achieving development goals, inequality is reduced both across and within countries. Investment in environmental technology and changes in tax structures lead to improved resource efficiency, reducing overall energy and resource use and improving environmental conditions over the longer term. Increased investment, financial incentives and changing perceptions make renewable energy more attractive. Consumption is oriented toward low material growth and lower resource and energy intensity. The combination of directed development of environmentally friendly technologies, a favorable outlook for renewable energy, institutions that can facilitate international cooperation, and relatively low energy demand results in relatively low challenges to mitigation. At the same time, the improvements in human well-being, along with strong and flexible global, regional, and national institutions imply low challenges to adaptation.
}

\subsection[The SSP1 mobility extension]{The SSP1 mobility extension: a narrative for the sustainable future}
\label{ss:results:ssp1-mob-development}
A particular \emph{vision} of a future sustainable mobility system is outlined in SSP1-MOB. Note that it is not comprehensive description of the myriad of elements composing the system, because it would not be feasible and too many sources of uncertainty would be introduced. Rather, it deals with travel patterns and which travel modes are most common, provided the necessary elements and configurations that make these patterns possible. The following is the narrated version of SSP1-MOB, describing the global situation of the mobility system in the year 2100:
%
\blockquote{\sffamily \textbf{SSP1-MOB narrative}\\Driven by an increasing level of awareness of the environmental and socio-economic impacts of the transportation system, the world has adopted a series of changes throughout the decades to reduce those. Technology-wise, vehicles have become more efficient and liquid hydrocarbon fuels are less carbon intensive and renewable (based on biofuels). More importantly, though, there has been a major shift in travel modes and, most importantly, total travel demand per capita has been reduced. However, the global absolute total demand has increased, due to economic growth and increases in the living standard of countries in Africa, Asia and South America.

An increase in urban density (1st), a change in land use patterns (2nd) and a de-centralisation of economic development hotspots (3rd) are at the core of the substantial change in travellers' needs and, thus, behaviour. More concentrated urban centres allow for a shortening of trip lengths, up to a point where cycling and walking are feasible alternatives. The size of cities, however, is kept below certain thresholds that permit, in principle, for a more livable and sustainable way of life. This indeed means that a de-centralisation process has taken place, from huge, unwieldy metropolis to medium-sized cities, allowing for (and requiring) a more horizontal economic and governance structure. Moreover, there has been a generalised backlash against single-use urban development, this is, there has been a move towards mixing cultural, residential, work, institutional and commercial uses of the built environment. This form of urban development acknowledges the limitations of single-use schemes, such as isolation and automobility dependence. These de-centralisation and mixing trends allow people to avoid the need for relocation to find a job or, for the same matter, commute long distances. Instead, short to medium distances and travel times are the norm when commuting, thus preventing the further need and use of automobiles.

High education levels, widespread access to fast internet and changes in consumption patterns also contribute to lower travel demand. Teleworking is increasingly adopted by companies and in other office-based jobs, allowing people to work from shared co-working facilities that are near home or to avoid commuting altogether. IT access also facilitates the spread of information systems to make car sharing, carpooling and, most importantly, intermodal travel possible (and feasible) --- users can access and use travel information to plan their routes easily, reducing both travel time and cost. Less travel intensive consumption patterns have emerged, such as a reduction in long distance tourism. Due to the fact that common consumption objects, such as food and other amenities can be found within the local area, the need to travel far away has been reduced.

When it comes to travel mode alternatives, most of the demand is supplied through public transport, be it in the form of passenger rail, buses or aviation. An increased and continuous heavy investment in public transport infrastructure (an extensive railway network, for example) has enabled fast, secure and low-carbon transport options for the majority of the population, which lives in more concentrated urban areas. High speed trains cover the demand for regional and national trips and even international trips (whenever the distance is not excessive). Regular electric trains and trams are the main mode of transport for interurban commuters and travellers. For less accessible areas, efficient, bio-fuelled, hybrid and electric buses are used. Fast and reliable hybrid or electric buses are also used for relatively short trips within cities. Aviation is used primarily for international travel, but the demand per capita has decreased. With regards to energy carriers for aviation, the main feasible alternative are highly energy-dense biofuels. In general, public transport poses to be a cheaper and more affordable option than automobile-based mobility, both for commuting and for leisure travel. Moreover, the perception of public mobility has changed and is now regarded as the best way to travel, due to higher levels of comfort, security and high reliability.

Even though public transport is the main and dominant travel mode, private mobility (automobility) is still a relevant mode in terms of total travel demand. The main reason for the long term survival of this mode is the higher degree of accessibility it provides, especially in remote, rural areas. While accessibility is kept at a high level thanks to this alternative, automobile (cars or two-wheelers) ownership rates are low. Car sharing is common and most urban communities benefit from reduced fleets thanks to carpooling, which is also commonly available and well accepted by the public. Within the specific mobility market that automobility now occupies, battery electric vehicles are the main car-based technological option used for short to medium ranged trips, such as commuting. Hydrogen-fuelled vehicles take the lead for longer trip distances. The cultural perception of the automobile as a status symbol has declined, in favour of the more environmentally sustainable public transport and slow modes like cycling and walking. Furthermore, the discourse around individual freedom that once legitimised automobility has been deeply challenged by the increasingly sustainability-concerned population.

Finally, slow travel modes such as walking and cycling have been adopted by many to cover the shortest inner-urban trips, especially amongst the youngest. Cycling lanes are an integral part of every urban area road network and public cycling facilities, such as secured parking stations or even public bike rental service schemes, are commonplace. Traffic regulation has been changed to prioritise and ensure the safety of both cyclists and pedestrians. Ample footpaths (sidewalks) provide not only the space for walking but also a more ``livable'' urban environment and pedestrian only streets and areas are also a common feature of urban neighbourhoods. This allows for a higher degree of community integration, which builds up social capital and increases social networks and safety nets.
}

The features and trends of the previous narrative, concerning mobility, are summarised in \tref{t:ssp1-mob-2100-narrative-thesis}. The intention is that the SSP1-MOB storyline fits within the assumptions of the \gls{IPCC} scenario as much as possible; in this regard, the SSP1 narrative provides with some preconditions for the mobility extension presented here that are considered out of scope from this study perspective, since they are not so directly related to mobility. The whole set of development trends that are required for the vision of SSP1-MOB to become a reality can be found in \textcite{vuuren2017_Energylanduse,oneill2017_roadsaheadNarratives}. Some of the most important prerequisites for a successful path to this vision are
%
\begin{itemize}
\item a wide societal support for sustainable development followed by
\item an actual change in investment patterns,
\item the decarbonisation and spread of the power grid,
\item the development of key technologies such as photo-voltaic systems, hydrogen fuel cell vehicles and electric batteries,
\item effective governance at national and international levels and
\item a change in cultural and ideological values towards the protection of the commons and away from the isolating individualisation that fosters the degradation of the environment and the social capital.
\end{itemize}
%
\begin{table}
\centering
\caption[SSP1-MOB qualitative variables]{Qualitative variables underlying to the SSP1-MOB narrative.}
\label{t:ssp1-mob-2100-narrative-thesis}
\footnotesize
\begin{tabular}{p{5cm}p{9cm}}
\toprule
Variable or feature & State \\ \midrule
\textbf{Development scenario} &  \\
Societal sustainability awareness & High \\
Total travel demand (Tpkm/yr) & Higher than the baseline (2017) \\
Travel demand per capita (Tpkm/yr) & Lower than the baseline (2017) \\ \addlinespace
\textbf{Land use (urban development)} &  \\
Urban density & High (higher than 2017 baseline) \\
Urban use patterns & Mixed-use development paradigm \\
Economic centralisation & Medium; cities are hotspots, but jobs are spread amongst them \\
City sizes & Medium; avoidance of megacities or (sub)urban sprawl \\\addlinespace
\textbf{Travel modes share} &  \\
Intermodal travel & Facilitated, high acceptancy and usage \\
Public transport (rail, bus, aviation) & Majority of demand supply; much higher than the baseline (2017) \\
Automobility (private vehicles) & Still relevant, but much lower than the baseline (2017) \\
Slow modes (walking and cycling) & Higher than the baseline (2017) \\\addlinespace
\textbf{Cultural perception} &  \\
Mobility & Accessibility, local in scale, slowed down, managed, reasonable travel time and reliability, integrated \\
Public transport & Public mobility as a reliable, comfortable, enjoyable and accessible service \\
Automobility & Automobility as a utility to serve a special need \\\addlinespace
\textbf{Public transport} &  \\
Reliability & High \\
Consumer cost & Low \\
Accessibility & High \\
Safety & High \\
Public transport infrastructure investments & High and continuous \\\addlinespace
\textbf{Automobility} &  \\
Reliability & High \\
Consumer cost & High \\
Accessibility & High (especially in rural or remote areas) \\
Safety & High (higher risk than public transport) \\
Automobility infrastructure investments (roads, fuel stations, etc.) & Low to medium; maintenance covers the majority of the investments; capacity is not increased \\\addlinespace
\textbf{Fuel technology} &  \\
Automobiles & Battery electric vehicles for short-medium ranged trips; hydrogen fuelled for long range \\
Rail & Full electrification of the network \\
Bus & Electric or hydrogen-fuelled \\
Aviation & Renewable biofuels \\ \bottomrule
\end{tabular}
\end{table}

\subsection{The sustainable mobility paradigm of SSP1-MOB}
\label{ss:results:ssp1-mob-paradigm}
The narrative presented in the previous section is meant to be a description of a \emph{sustainable mobility paradigm} of the future. It is important to stress that this paradigm is indeed radically different from, and not solely a continuation of, the current mobility system. The fundamental difference lies not in the transport technologies available (there is no way to forecast or backcast a possible technological breakthrough of the future), but in how mobility is understood and how it is related to lifestyles, how it is embedded in cultural frames and, in turn, how does it shape lifestyles and the infrastructure supporting them.

The SSP1-MOB paradigm is built around an important, but often overseen\footnote{Many institutions and researchers envision sustainable mobility through the lenses of technological improvements and efficiency gains, deeming them sufficient to tackle the wide range of hazardous impacts of the current transportation system.}, aspect of sustainable mobility: the overall \emph{reduction in travel demand per capita}. This is, one of the keys to a sustainable future transport system is actually a cut in mobility, which is rooted in (a) land use patterns and (b) urban and economic organisation. Extending from what \textcite{moriarty2008_Lowmobilityfuture} argue in relation to a low-mobility future, by having people's daily activities (jobs, facilities and entertainment areas) closer the need for mobility is reduced. In turn, a lower need for travel ends up shaping the lifestyle in which the mobility paradigm is embedded, by reinforcing the urban ``re-localisation'' pattern.

Therefore, in the future of SSP1-MOB, mobility is conceived no longer as a right that must be guaranteed nor as a ``fundamental human desire'' to be pursued. Mobility is understood for what it really is: a service/activity whose demand is \emph{derived} from the activities that are performed at the end of the trips themselves. Following this mental framing of mobility, together with the (assumed) wide acknowledging of the impacts derived from this demand, it is both the activities that drive mobility and the transport system itself that are conceived differently: the era of private, pervasive, cheap, long-distance travel has been left behind in SSP1-MOB. Lifestyles have been accommodated to (a) avoid the need of (high) mobility, (b) take advantage of extensive and highly reliable public transport networks and (c) embrace active transportation (cycling and walking) as soon as the circumstances allow for it.

Urban planning is not the only cause of reduced travel demand and automobility independence. Low-mobility social and entertainment activities are dominant in SSP1-MOB, in clear opposition to, for example, the current trends of increased intercontinental tourism (which is taken by many as a yearly option for their holidays). Local, cultural or social activities are engaged by the majority of the population, which enhances the social bonds of the community, further decreasing the desire of individuals to ``flee'' from dense urban areas to be ``left on their own''.

Finally, with regards to the direct environmental impacts of transportation, the SSP1-MOB paradigm also contains insights into what a sustainable form of mobility entails. Air pollution is kept to the absolute minimum in the SSP1-MOB vision: the only carbon-intensive means of transportation are aviation and biofuelled vehicles. However, these are assumed to be rare and account for a small proportion of the total mobility demand. In this regard, the extensive electrification of the transport network (both train and electric buses and cars) has meant a true technological revolution. Other hazardous impacts of mobility, such as accidents and noise pollution, are kept at very low levels too, since the automobility demand has vastly reduced.