%%%% Abstract page %%%%
\cleardoublepage % Clear the previous page
\phantomsection % Create Hyperref anchor
\addcontentsline{toc}{chapter}{Abstract} % Add the abstract to the Table of Contents

\chapter*{Abstract} % Abstract section, unnumbered ("*")
The thesis explores a possible sustainable future of mobility and the transition pathway to it, focusing on the socio-cultural dimensions that shape and drive the way mobility is understood. Goal-driven, transition-oriented policy recommendations are provided as the main result, derived from a combined backcasting and forecasting methodology framework. The successful combination of backcasting and Causal Loop Diagrams is achieved by homogenising the outcomes of each assessment through the logic of the Multi-Level Perspective of transitions theory.

The research highlights that reinforcing feedback mechanisms and a deeply embedded culture of automobility are behind the enormous inertia and resilience of the current mobility system. If a transition to a sustainable mobility future is to happen, the insights gained from this study point to a necessary shift in cultural trends. The discourses of unrestricted individual freedom, private property and materialistic cultures that legitimise automobility must be challenged.

The thesis proves that the Multi-Level Perspective on transitions provides with a narrative capable of integrating results from inherently different approaches to future studies. The methodological framework developed in the study is generalisable and useful for situations where a normative goal in the distant future is pursued, while accounting for the reasons behind policy resistance in the current system configuration.

\cleardoublepage % Clear the page for other sections (front matter or main matter)
