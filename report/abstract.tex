%%%% Abstract page %%%%
\cleardoublepage % Clear the previous page
\phantomsection % Create Hyperref anchor
\addcontentsline{toc}{section}{Abstract} % Add the abstract to the Table of Contents

\section*{Abstract} % Abstract section, unnumbered ("*")
Today the world is consuming almost \SI{70000}{\mega\tonne} of apparel every year, this accounts for $2\%$ of the world’s GDP and presents a large threat to the environment \parencite{icac2013}. One way to reduce the environmental impact is to change to another type of fabric than the traditional ones. This report is investigating if Swedish viscose could be that type of fabric, it is labelled as a natural product since it is produced from wood pulp. The LCA is conducted for two of T-shirts made from Swedish viscose and Asian cotton. The functional unit is a cotton or viscose T-shirt of \SI{200}{\g} during its whole life cycle, including the disposal phase, used in Sweden during 2 years, washed once every two weeks (for a total of 52 wash cycles). The system boundary for this study is cradle-to-grave. The impact categories that are assessed for the study are \textit{freshwater eutrophication}, \textit{terrestrial ecotoxicity} and \textit{water depletion}. The LCA study shows that the viscose T-shirt has the lowest overall impact compared to the cotton T-shirt. A sensitivity analysis is conducted by remove the tumble dryer in the use phase. The lack of LCI data for making a T-shirt with viscose fabric could reduce the credibility of the results, future LCA studies are needed with data from the production phases of a viscose T-shirt.

\cleardoublepage % Clear the page for other sections (front matter or main matter)
